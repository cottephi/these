\chapter{L'expérience \texorpdfstring{DU$\nu$E}{DUNE} et son projet de R\&D~: \texorpdfstring{protoDU$\nu$E}{protoDUNE}}
    \chapterprecishere{
        ``Potentielle citation sans aucun rapport avec le sujet"\par\raggedleft--- \textup{Personne inconnue}, contexte à déterminer
    }
    
    Ce chapitre commence par introduire les principales expériences d'oscillation de neutrinos d'accélérateurs à long ligne de base, dont \gls{dune} fait parti, en expliquant leurs points forts et leurs points faibles. Ensuite, une section est dédiée à l'expérience \gls{dune} elle même, puis une autre section est dédiée au projet \gls{wa105} faisant parti des expériences de prototypage de \gls{dune}.
        
    \section{DU$\nu$E, une expériences d'oscillation de neutrinos d'accélérateurs à long ligne de base}
    
	    %Généralités sur DUNE: où, quand, pourquoi.
        \gls{dune} est une expérience d'oscillation des neutrinos à longue ligne de base, qui enverra un faisceau de neutrino depuis le Fermilab (Illinois) vers l'installation de recherche souterraine de Sanford (Dakota du sud)(voir \autoref{fig::dune}). 
        
        Le faisceau de neutrinos, détaillé en \autoref{sec::faisceau}, sera créé au Fermilab. Après avoir parcouru \SI{300}{\meter}, il traversera un détecteur proche servant à le caractérisé en détail (flux, composition, énergies...) puis parcourra une ligne de base de \SI{1300}{\kilo\meter} à travers la croûte terrestre avant d'atteindre le détecteur lointain à Sanford, à \SI{1475}{\meter} de profondeur, on s'effectuera la plus grande partie des mesures de physique. Ce détecteur lointain sera composé de quatre modules de \glspl{tpc} chacune remplie de $\SI{10}{\kilo\tonne}$ d'argon liquide. Cette thèse porte sur le prototypage d'une versions de cette technologie permettant l'amplification des charges dans une mince phase d'argon gazeux. Les détails sont apportés en \autoref{sec::lartpc}. Les objectifs de \gls{dune}, détaillés en \autoref{sec::dune_pheno}, sont multiples. Mais \gls{dune} a été pensée essentiellement pour permettre la détermination de la hiérarchie de masse en exploitant les effets de matière sur la probabilité de changement de saveur $P(\nu_{mu} \to \nu_e)$ (voir \autoref{sec::hierarchy}) ainsi que la mesure de la phase de violation de CP, puisque la distribution des énergies des neutrinos du faisceau permettra de couvrir deux maxima locaux de probabilité de changement de saveur (voir \autoref{sec::CP_violation}), permettant de séparer les effets de matière des effets de la phase de violation CP. 
        
        Les constructions civiles du détecteur lointain, à Sanford, ont commencé en 2017 avec le début des excavations. Elles seront suivies par les installations cryogéniques, nécessaires aux opérations utilisant de l'argon liquide. Les modules du détecteur lointain seront ensuite progressivement installés, la mise en service et les premières prises de données du premier module étant prévues pour 2024\cite{Acciarri2016}. Le dernier module, qui pourra potentiellement utiliser la technologie à \gls{dlartpc} étudiée dans cette thèse, devrait commencer à prendre des données vers 2028.
        
        En parallèle, l'injecteur principale du Fermilab, qui a fourni les neutrinos des expériences NO$\nu$A et MINOS, sera mis à niveau par le projet PIP-II qui permettra d'atteindre une puissance de \SI{1.2}{\mega\watt} d'ici 2026, ce qui en fera le faisceau de neutrinos le plus intense de la planète. Une autre mise à niveau, PIP-III, est prévue pour 2030 et permettra d'atteindre \SI{2.4}{\mega\watt}.
        
        Les expériences à longue ligne de base sont caractérisées par le fait de chercher à maximiser les probabilités de changement de saveur des neutrinos. Comme nous l'avons vu en \autoref{sec::oscillations}, ces probabilités oscillent suivant un angle dépendant du ration $L/E$, où $L$ est la distance parcourue par les neutrinos (ligne de base) et $E$ leur énergie. Il est possible d'optimiser le ratio $L/E$ afin d'atteindre un maximum de probabilité. En plus de cela, l'expérience \gls{dune} cherche à se placer à une ligne de base suffisamment importante pour que les effets de matière permettent la mesure de la hiérarchie de masse. La ligne de base choisie, de \SI{1300}{\kilo\meter}, doit être accompagnée d'une énergie autour de \SI{2.5}{\giga\electronvolt} (voir \autoref{fig::3flavors_oscillations}) afin d'atteindre le premier maximum d'oscillation. Il est également possible de couvrir le second maximum, situé autour de \SI{0.9}{\giga\electronvolt}, en générant un faisceau de neutrino à large spectre. Les neutrinos du faisceau que détectera \gls{dune} auront une énergie comprise entre \numprint{0.5} et \SI{5}{\giga\electronvolt}, couvrant ainsi les deux maxima.
        
        \subsection{Qu'est ce qu'une expérience d'oscillation de neutrinos d'accélérateurs à longue ligne de base?}
            
            Les expériences d'oscillation de neutrinos d'accélérateurs à longue ligne de base cherchent à mesurer la probabilité d'oscillation $P(\nu_{\mu}\to \nu_e)$ en fonction de l'énergie du neutrino incident en détectant les produits de réaction des neutrinos d'un faisceau initialement composé uniquement de neutrinos muoniques. Ce faisant, ces expériences ont accès aux angles de mélanges des secteurs atmosphériques et réacteurs de la matrice PMNS décrite en \autoref{sec::PMNS}, ainsi qu'à la phase de violation de CP et aux valeurs absolues des différences de masse carré de ces mêmes secteurs. Si les effets de matière sont suffisamment importants, elles peuvent sonder la hiérarchie de masse.
            
            Le détecteur doit donc être capable de différencier une réaction de neutrino muonique d'une réaction de neutrino électronique, ainsi que de mesurer précisément l'énergie mise en jeu dans ces réactions. Le faisceau de neutrino, dont le principe est décrit en \autoref{sec::faisceau}, va d'abord traverser un détecteur placé proche de la source, typiquement à quelques centaines de mètres, quand la probabilité d'oscillation est encore nulle. Les caractéristiques importantes du faisceau y sont analysées, comme le flux et le spectre en énergie. Le faisceau parcours ensuite une grande distance à travers la croûte terrestre jusqu'au détecteur lointain. 
            
            La possibilité de comparer les observations avant et après oscillation d'une même source de neutrinos est une plus-value des expériences à longue ligne de base utilisant des faisceaux ou des réacteurs. Une plus-value du faisceau est son ajustabilité en flux et en énergie. Le spectre en énergie peut être choisie précisément, dans une gamme d'énergie allant de \SI{0.5}{\giga\electronvolt} à \SI{100}{\giga\electronvolt}, et il est possible de choisir entre un faisceau de neutrino ou d'anti-neutrino. Bien que la mesure de l'oscillation des neutrinos seule suffise théoriquement à mesurer la phase de violation de CP et la hiérarchie de masse (voir \autoref{sec::3flavor_matter}), la comparaison à l'oscillation d'anti-neutrino permet de gagner en précision.    
                    
        \subsection{Expériences passées, actuelles et futures}
        
            Les expériences à longues ligne de base avec neutrinos d'accélérateurs sont apparues dans la fin des années 90. Les premières et deuxièmes générations ont confirmé les observations indiquant l'existence des oscillations des neutrinos, et ont effectué des mesures de précisions sur les angles $\theta_{13}$ et $\theta_{23}$ ainsi que sur la différence de masse carré $\Delta m^2_{32}$\cite{pdg2018}. Les expériences de première génération étaient, dans l'ordre chronologique, K2K\cite{Collaboration2006a} (Japon, 1999--2004), MINOS\cite{Collaboration2014} (USA, 2005-2011) et OPERA\cite{Agafonova2018} (Europe, 2008--2012). La seconde génération, encore en activité, sont les expériences NO$\nu$A\cite{Adamson2016} (USA, 2014--) et T2K\cite{Abe2018} (Japon, 2010--). Les expériences de la troisième génération, en développement, sont Hyper-Kamiokande\cite{HK2018} prévue au Japon pour  
            %DATE ET NOM de Tokai to HK?
            et \gls{dune}\cite{Acciarri2016}, prévue aux USA pour 2026. Ces dernières ont pour but la mesure de la phase de violation de CP, dont une valeur différente de $0$ et $\pi$ a été annoncée avec un niveau de confiance de $2\;\sigma$\cite{t2k-cp} par T2K en 2011, et la hiérarchie de masse.
            
            \subsubsection{Les expériences de première génération}
            
            \paragraph{L'expérience K2K\cite{Collaboration2006a}:} C'était la première expérience de faisceau à longue ligne de base, construite dans le but de vérifier les résultats de Super-Kamiokande sur les oscillations des neutrinos atmosphériques. Un faisceau de neutrinos muoniques produit par un synchrotron de proton de $\SI{12}{\giga\electronvolt}$ au KEK, Tsukuba, Japon, parcourait $\SI{300}{\meter}$ jusqu'à un détecteur proche Cherenkov à eau de $\SI{1}{\kilo\tonne}$. Le faisceau traversait ensuite $\SI{250}{\kilo\meter}$ jusqu'à Super-Kamiokande, un détecteur Cherenkov à eau de $\SI{50}{\kilo\tonne}$. K2K a permit de montrer à un niveau de confiance de $4.3\;\sigma$ que les neutrinos muonique disparaissait, apportant ainsi une confirmation de la théorie des oscillations des neutrinos. K2K a également fournit une mesure de la différence des masses carrées de $|\Delta m^2_{32}|=2.8^{+0.7}_{-0.9}\times\SI{e-3}{\electronvolt\squared}$.
            
            \paragraph{MINOS\cite{Collaboration2014}:} L'expérience MINOS avait pour but de mesurer avec ue grande précision le secteur atmosphérique. Elle observait la disparition des neutrinos muoniques de $\SI{3}{\giga\electronvolt}$ produits par la ligne de faisceau NuMi, au fermilab, avec une ligne de base de \SI{735}{\kilo\meter}, le détecteur lointain, un calorimètre à échantillonage en acier scintillateur de $\SI{5.4}{\kilo\tonne}$, étant situé dans le Minnesota du nord. Elle a mesurée la différences de masse carré $\Delta m_{32}^2 = 2.35^{+0.11}_{-0.08}\times\SI{e-3}{\electronvolt\squared\per c^4}$ et l'angle $\sin^2(2\theta_{23}) > 0.91$ (limite de confiance de 90\%). 
            
            \paragraph{OPERA\cite{Agafonova2018}:} L'expérience OPERA diffère légèrement des autres expériences à longue ligne de base d'accélérateur par le fait qu'elle cherchait à observer l'apparition de neutrino tauique. Elle utilisait pour ce faire un trajectographe à émulsion, constitué de \numprint{150000} briques de films photographiques espacés par des feuilles de plomb, arrangées en murs parallèles espacés par des compteurs en scintillateurs plastiques. Les neutrinos, produits par le SPS du CERN, avaient une énergie entre 5 et $\SI{25}{\giga\electronvolt}$ et traversait $\SI{732}{\kilo\meter}$ jusqu'au laboratoire du Gran Sasso. Le rapport $L/E$ n'était alors pas au maximum d'oscillation, qui correspond à une énergie de $\SI{1.6}{\giga\electronvolt}$ pour cette ligne de base, mais le seuil de production du $\tau$ de $\SI{3.5}{\giga\electronvolt}$ était dépassé. Au total, 5 neutrinos tauiques ont été observés.
            
            \paragraph{ICARUS\cite{icarus}:} ICARUS est la première expérience de neutrino utilisant de l'argon liquide comme milieu d'interaction dans son détecteur. Elle était au Gran Sasso de 2004 à 2012 mais n'a reçu des neutrinos du SPS du CERN qu'à partir de 2010. Comme il se trouvait au Gran Sasso, sa ligne de base était la même que celle d'OPERA, et le ratio $L/E$ ne correspondait pas à un maximum d'oscillation. ICARUS n'a donc pas produit de résultats majeurs en physique des neutrinos en dehors d'une limite sur la masse de neutrinos stériles $\Delta m^2 > \SI{0.01}{\electronvolt\squared}$. Mais elle a démontré que le principe d'une chambre à projection temporelle de plusieurs centaines de kilo tonnes utilisant de l'argon liquide est adaptée à l'observation d'événements rares comme les interactions neutrinos. 
            
            \subsubsection{Les expériences de seconde génération}
            
            Les deux expériences de seconde génération, NO$\nu$A aux États-Unis et T2K au Japon, sont complémentaires : elles ont les même objectifs mais utilisent des technologies différentes.  Elles visent toutes les deux à mesurer $|\Delta m_{32}^2|$ et $\sin^2{\theta_{23}}$. Elles cherchent également à sonder la phase de violation CP et la hiérarchie de masse. Les différences principales entre les deux expériences sont les technologies de détection (détecteur Cerenkov pour T2K et scintilqteur liauide pour NO$\nu$A) et la lignes de base, qui est un paramètre important pour la détermination de la hiérarchie de masse : \SI{295}{\kilo\meter} pour T2K et \SI{810}{\kilo\meter} pour NO$\nu$A. Les deux expériences ont leurs détecteurs proches et lointains hors axe ($2.5^{\circ}$ pour T2K et $\SI{14}{\milli\radian}$ pour NO$\nu$A), afin de restreindre le spectre en énergie des neutrinos.
            
            \paragraph{NO$\nu$A\cite{Adamson2016}:} Comme MINOS, NO$\nu$A détecte des neutrinos créés au fermilab dans la ligne de faisceau NuMI. Le détecteur lointain a été placé le plus loin possible (tout en restant aux États-Unis) afin de favoriser la sensibilité à la hiérarchie de masse. Pour rappel, plus la distance parcourue dans la matière par les neutrinos est grande, plus la hiérarchie de masse est facile à déterminer. L'énergie des neutrinos reçus est de \SI{2}{\giga\electronvolt}, correspondant au premier pic de probabilité de disparition des neutrinos muoniques. Sont détecteur lointain pèse \SI{14}{\kilo\tonne} et est constitué de \numprint{344064} cellules PVC de $\SI{15}{\meter}\times\SI{4}{\centi\meter}\times\SI{6}{\centi\meter}$ remplies de scintillateur liquide, le tout relié à des photodiode à avalanche pour amplifier le signal. Le détecteur proche est identique mais plus petit. NO$\nu$A prend des données depuis 2014 et a publié un premier papier en 2016\cite{Adamson2016}.
            
            \paragraph{T2K\cite{Abe2018}:} T2K utilise le même détecteur lointain que K2K, à savoir Super-Kamiokande. Il reçoit des neutrinos du synchrotron de \SI{145}{\kilo\watt} J-PARK, avec une énergie piquée autour de \SI{0.6}{\giga\electronvolt}, correspondant au premier maximum de probabilité de disparition des neutrinos muoniques à la ligne de base de T2K de \SI{295}{\kilo\meter}. Les premiers neutrinos de faisceaux sont arrivés en 2010, suivi d'un papier en 2011\cite{Abe2011}.
            
            \subsubsection{Les futurs expériences}
            
            Les deux futurs expériences à longue ligne de base d'accélérateur sont Hyper-Kamikande, prévue au Japon aux même sites que l'actuelle T2K, et \gls{dune}, aux États-Unis. Elles auront toutes deux pour objectif la mesure de la phase de violation CP et la détermination de la hiérarchie de masse, mais utiliseront deux approches différentes. Hyper-Kamiokande favorisera une ligne de base plus petite, améliorant la sensibilité à la phase de violation CP au détriment de la précision sur la hiérarchie de masse. \gls{dune} utilisera une ligne de base plus grande et un spectre en énergie assez large pour couvrir deux maxima de probabilité d'oscillation, permettant de sonder avec une bonne précision à la fois la hiérarchie de masse et la phase de violation CP.
            
            \paragraph{Hyper-Kamiokande\cite{hyper-k}:} Hyper-Kamiokande aura la même ligne de base de T2K. Sont détecteur lointain est prévu de peser \SI{560}{\kilo\tonne} fiducièles. La même technologie que T2K sera utilisé, à savoir un détecteur Cerenkov à eau. Il sera désaxé de la même manière que T2K, de $2.5^{\circ}$, et recevra des neutrinos de même énergie que T2K. Le faisceau aura cependant une puissance supérieur, de \SI{1.3}{\mega\watt}. L'expérience a été approuvée en 2018 et la construction débutera en 2020.
            
            \paragraph{DU$\nu$E\cite{Acciarri2016}:} \gls{dune} est décrit en détail dans les sections qui suivent. Elle aura une ligne de base de \SI{1300}{\kilo\tonne}, les neutrinos arrivant sur sont détecteurs lointain auront une énergie entre comprise $\SI{0.5}{\giga\electronvolt}$ et $\SI{5}{\giga\electronvolt}$, contenant les deux maxima de probabilité d'oscillation. Ses détecteurs lointains seront des \gls{tpc} à argon liquide avec une masse totale de \SI{40}{\kilo\tonne} répartie en 4 modules. L'expérience a été approuvée en 2015 et les premières prises de données devraient débuter en 2026.
            
            \paragraph{ESS$\nu$SB\cite{Acciarri2016}:} Une troisième expérience, ESS$\nu$SB, est en discussion\cite{essnusb}. Il s'agirait d'utiliser le faisceau de la future European Spallation Source, en Europe du nord, afin de délivrer un faisceau de \SI{10}{\mega\watt} sur un détecteur lointain de \SI{0.5}{\mega\tonne} utilisant la technologie Cerenkov à eau à une ligne de base de \SI{500}{\kilo\meter}. Sont principal objectif serait la mesure de la phase de violation CP.
    
    \section{L'expérience \texorpdfstring{DU$\nu$E}{DUNE}}\label{sec::dune}
    
        \subsection{Introduction}
            \cite{Acciarri2016}\\
            Depuis la découverte des oscillations des neutrinos par SNO et Kamiakande, il est clair que ce domaine de la physique des particules présente des aspects au delà du modèle standard, qui ne prédit pas de masse pour les neutrinos. De ce fait, c'est un domaine qui suscite l'intérêt de la communauté scientifique dans le monde entier, intérêt qui n'a fait qu'augmenter depuis le prix Nobel de 2015 pour la découverte de ces oscillations. L'annonce par le ministère de l'énergie des États-Unis de sa volonté d'améliorer le dispositif de faisceau du Fermilab avec la mise à niveau PIP-2 a donné une opportunité à cette communauté de disposer d'un faisceau de $\SI{1.2}{\mega\watt}$ pour une expérience d'oscillation de neutrino à longue ligne de base d'accélérateur. Le projet \gls{dune} a alors été imaginé, pour aboutir en 2016 à un rapport d'étude conceptuel de 4 volumes. Le projet regroupe aujourd'hui un millier de collaborateurs, répartis dans 175 institutions et 32 pays.
            
            \gls{dune} emploiera la technologie de \gls{tpc} à argon liquide, déjà utilisée avec succès par ICARUS, pour ses \SI{40}{\kilo\tonne} du détecteur lointain. Elle aura également la plus longue ligne de base à ce jour pour les expériences à longue ligne de base d'accélérateur, favorisant ainsi la découverte de la hiérarchie de masse. Le spectre en énergie des neutrinos émis par le faisceau couvrira deux maxima de probabilité d'oscillation, permettant d'accéder plus facilement à la phase de violation CP, qui est plus visible sur le second pic. Après une dizaine d'années d'opération, la précision attendue sur la phase violation CP est de 5(10)\% si $\delta_{CP}=0(90)^{\circ}$. La probabilité de déterminer la hiérarchie de masse est quant à elle de 99.9996\%\cite{Acciarri2016} ou plus.

        \subsection{Objectifs scientifiques}\label{sec::dune_pheno}
        
            \paragraph{Quelle est la valeur de la phase de violation CP dans la matrice \gls{pmns}?} : Andrej Sakharov a montré en 1967\cite{Sakharov1967} que l'absence apparente d'antimatière dans l'univers pouvait être expliquée par la violation de la symétrie CP en physique des particules (voir \autoref{sec::CP_violation}). Une telle asymétrie implique que matière et antimatière peuvent se comporter différemment dans certains processus physique, expliquant ainsi l'absence de matière dans l'univers. Une telle violation CP existe bien dans le modèle standard, dans la matrice CKM régissant le comportement des quarks. Mais les observations cosmologiques actuelles\cite{Bari2012} indiquent que cette violation n'est pas suffisante pour expliquer la quantité de matière présente dans l'univers observable, en supposant qu'autant de matière et d'antimatière ait été formé lors du Big Bang. Il est donc nécessaire de trouver une nouvelle physique, au delà du modèle standard, qui pourrait elle aussi présenter une violation CP. Il se trouve que les oscillations des neutrinos sont un phénomène au delà du modèle standard, puisqu'elles impliquent que les neutrinos ont des masses non nulles. Et une violation CP est possible dans la théorie des oscillations. Une étude de 2018\cite{Bucella2018} a montré que, sous certaines hypothèses, une phase comprise entre $\frac{3\pi}{4}$ et $\pi$ pourrait suffire à expliquer l'asymétrie matière antimatière.
            
            \paragraph{Quelle est la hiérarchie de masse des neutrinos?} : un des paramètres encore inconnue des oscillations des neutrinos est le signe de différence de masse carrée entre l'état propre de masse $\nu_1$ et l'état propre de masse $\nu_3$. De ce fait, nous ne savons pas quel état propre est le plus lourd des deux.
            
            \paragraph{Mesures de précisions des paramètres gouvernant les oscillations des neutrinos} : Il est toujours bon d'avoir des redondances dans les mesures de paramètres physique. \gls{dune} va donc mesurer les angles et différences de masses déjà connues, avec une précision meilleure. Plus particulièrement, \gls{dune} tentera de déterminer l'octant de l'angle $\theta_{23}$, qui est pour le moment estimé autour de $45^{\circ}$, mais dont les erreurs de mesure lui permettent d'être au dessus ou en dessous de cette valeur\cite{citation_needed}.
            
            \paragraph{Le proton se désintègre-t-il?} : Les théories de grande unification\cite{Pati1973} montrent que les interactions fondamentales du modèle standard (interaction électro-faible et interaction forte) peuvent être unifiées dans une interaction unique à très haute énergie. Si cela est le cas, le nombre baryonique peut ne pas être conservé, et le proton peut alors avoir un temps de vie très grand mais fini. Une expérience pouvant observer un nombre important de proton avec un minimum de bruit de fond peut chercher à détecter les produits d'une désintégration de proton. Les expériences de neutrinos comme \gls{dune}, qui nécessitent des kilo-tonnes de matériau réactif dense tout en étant très bien isolées des rayons cosmiques et des désintégrations nucléaire sont idéales pour ces observations, d'autant que les produits de désintégrations de proton sont bien discernables de produits d'interaction de neutrino.
            
            \paragraph{Autres objectifs} :
                \gls{dune} a également des objectifs scientifiques secondaires:
                \begin{itemize}
                    \item Détection d'interactions non standards : des interactions non prédites par le modèle standard (SUSY, théorie des cordes...)
                    \item Couplages au neutrinos stériles : il ne peut existé que 3 saveurs de neutrinos capables de se coupler au bosons Z\cite{Olive2016}. Mais il peut exister d'autre neutrinos dit stériles, qui n'interagissent pas, et qui pourrait être des candidats à la matière noire. Un manque de neutrinos ne pouvant être imputées aux oscillations entre les trois saveurs connues peut être un indice d'une oscillation vers un neutrino stérile.
                    \item Étude du neutrinos tau : seule OPERA s'est intéressée aux oscillations vers le neutrino tau pour le moment, et avec peu de statistiques. Une seconde mesure permettra de valider ses résultats.
                    \item Détection de neutrinos de supernovae : les supernovae sont des phénomènes encore mal compris. Détecter des neutrinos issues de leur explosion permettrait de tester les théories à leur sujet.
                    \item Mesures de précision des interaction neutrinos : les interactions de neutrinos étant naturellement rares, les différentes sections efficaces de ces dernières ne sont pas mesurées avec une précisions. Les interactions de neutrinos peuvent également être utilisées pour sonder la structure des nucléons.
                    \item Détection de matière noire : \gls{dune} cherchera également à détecter un spectre continue de neutrino à haute énergie, qui serait un indice pour la désintégration de matière noire en neutrinos dans le coeur du soleil\cite{Rott2017}. 
                \end{itemize}
            
        \subsection{Description de l'expérience}
        
            \subsubsection{Le faisceau de neutrino}\label{sec::faisceau}
        
            Il existe trois principaux types de faisceau de neutrino :
            \begin{itemize}
                \item les faisceaux $\beta$, qui récupèrent les neutrinos issues de désintégrations $\beta$,
                \item les usines à neutrinos, qui récupèrent les neutrinos issues de la désintégrations de muons
                \item les faisceaux mésoniques, qui récupèrent les neutrinos issues de la désintégrations de méson.
            \end{itemize}
            Seul la dernière de ces trois technologies a déjà été utilisé pour des expériences, les deux premières étant encore en phase de prototypage. L'expérience \gls{dune} utilisera un faisceau mésonique, aussi nous nous concentrerons sur cette technologie dans la suite du texte. Pour plus d'information sur les faisceaux $\beta$, le lecteur peut consulter l'article~\cite{Wildner2012}, et l'article~\cite{Bogomilov2014} pour plus d'information sur les usines à neutrinos.
            
            %Le faisceau $\beta$, le plus récent des trois, est encore en phase de prototypage. Nous n'en parlons que rapidement, par soucis d'exhaustivité, car ce n'est pas la technologie employée dans \gls{dune}. Plus de détails sont consultables dans cet article~\cite{Wildner2012}. Il s'agit de produire et stocker des noyaux radioactifs se désintégrant via la désintégration $\beta$ puis de les collimater pour de les laisser se désintégrer et ainsi produire un faisceau de (anti)neutrino électronique. L'avantage de cette technologie est d'avoir un spectre en énergie très restreint et ajustable, possiblement de basse énergie permettant de placer un détecteur relativement proche de la source pour correspondre à un maximum de probabilité d'oscillation, limitant ainsi les effets de matière sur les oscillations. Ceci permet d'être plus sensible à la violation CP qu'avec une expérience à longue ligne de base.
            
            %La seconde technologie de faisceau de neutrino est l'usine à neutrino. Comme pour le faisceau $\beta$, elle est en cours de développement et ne sera pas utilisé dans \gls{dune}, nous n'en parlons donc que rapidement, plus de détails se trouvant ici~\cite{Bogomilov2014}. Il s'agit de produire des mésons en projetant des protons sur une cible, de laisser ces mésons se désintégrer en muons et anti-muons qui sont stockés dans un anneau afin de les laisser se désintégrer à leur tour en neutrinos et anti-neutrinos électroniques et muoniques. L'avantage de cette technologie est de produire des faisceaux composés précisément de 50\% de neutrinos muonique et de 50\% de neutrinos électronique, contrairement à la technologie actuelle qui, comme décrit plus bas, produit des faisceaux avec environ 99\% de neutrinos muoniques et environ 1\% de contamination de neutrinos électroniques. En effet la connaissance précise de la composition en saveurs du flux de neutrino est primordiale pour effectuer des mesures d'oscillation de saveurs.
            
            %La troisième technologie, schématisée en \autoref{fig::faisceau}, est celle qui est utilisée actuellement dans les expériences d'oscillation de neutrinos d'accélérateurs à longue ligne de base, et qui est prévue pour \gls{dune}. 
            La \autoref{fig::faisceau} est un schéma de la création d'un faisceau de neutrinos par désintégration de mésons. Il s'agit de produire des mésons (essentiellement des pions et des kaons) en envoyant des protons sur une cible fixe, généralement du graphite. Ces mésons sont ensuite focalisés par une ou plusieurs cornes magnétiques, qui permettent également d'enlever les mésons chargés positivement ou négativement suivant que l'on cherche à produire un faisceau de neutrinos ou d'anti-neutrinos. Les mésons restant sont alors envoyés dans un long tuyau (pouvant être de l'ordre du kilomètre) pour s'y désintégrer en muons et neutrinos muoniques, avec une faible contaminations de neutrinos électronique due au canal de désintégration rare du kaon $K$ $K^+ \to e^+ \pi^0 \nu_e$. Si l'énergie est suffisante, il peut également y avoir une faible contamination en neutrinos tauiques due à la désintégrations de mésons $D_s$. Les mésons restant ainsi que les produits de désintégrations autres que les neutrinos sont arrêtés par une couche épaisse, en acier dans l'expérience \gls{dune}. Il est également possible de se passer de tuyau de désintégration et d'envoyer directement les mésons dans la roche ou le béton, afin qu'ils interagissent avant de se désintégrer. Les neutrinos viendront alors des produits de ces interactions à court temps de vie. Le faisceau résultant sera bien moins intense, mais sera composé d'autant de neutrinos que d'anti-neutrinos, électroniques et muoniques, ainsi que d'une fraction non négligeable de neutrinos et d'anti-neutrinos tauiques. Cette technique, appelée \textit{beam dump}, n'est pas utilisée dans \gls{dune}, qui cherche à avoir des neutrinos d'une seule saveur dans son faisceau.
            
            Les articles \cite{Levy2010,McDonald2001,Itow2001} montrent que le spectre en énergie attendue des neutrinos dépend de l'angle entre la direction de propagation du neutrino et l'angle principal du faisceau. Positionner un détecteur de manière légèrement désaxé (\SI{2.5}{\degree} pour l'expérience T2K par exemple) permet non seulement de choisir le maximum du spectre à une énergie correspondant à un maximum de probabilité d'oscillation, mais aussi d'avoir un spectre plus étroit. Dans \gls{dune}, l'idée est d'avoir un spectre assez large pour couvrir deux maxima de probabilité, aussi les détecteurs proche et lointain seront situés sur l'axe du faisceau.
            
            \paragraph{Le faisceau de proton de DU$\nu$E:} Le faisceau de proton servant à produire les neutrinos pour \gls{dune} sera créé par le complexe d'accélérateur du Fermilab, qui aura été mis à niveau pour fournir une puissance de $\SI{1.2}{\mega\watt}$ pendant les cinq premières année. Une mise à niveau pourra monter cette puissance à $\SI{2.4}{\mega\watt}$. Il fonctionnera par cycle produisant chacun \numprint{1.0e-5} protons de $\SI{60}{\giga\electronvolt}$ (cycle de \SI{0.7}{\second})  à $\SI{120}{\giga\electronvolt}$ (cycle de \SI{1.2}{\second}). 
            
            \paragraph{La cible:} Les protons iront frapper une cible longue de $\SI{95}{\centi\meter}$ faite de 47 segments de graphites, où 85\;\% des protons interagiront pour produire en majorité des pions et des kaons. 
            
            \paragraph{Les cornes magnétiques:} Une corne magnétique consiste en deux conducteurs de symétrie axiale, un intérieur et un extérieur. Un courant entre par le conducteur intérieur et ressort par le conducteur extérieur, créant un champ magnétique entre les deux conducteurs. Les particules d'un signe de charge électrique donné produites dans la cible sont libres de traverser le conducteur intérieur -- qui doit être assez fin pour ne pas en absorber trop et assez épais pour supporter les forces magnétiques -- et seront ensuite déviées pour être parallèles à l'axe de la corne par le champ magnétique. Les particules de signe opposé seront déviées vers l'extérieur de la corne et s'en échapperont. Changer le sens du courant change le signe qui focalisé, et permet de choisir entre neutrinos et antineutrinos. \gls{dune} sera muni de deux cornes magnétique dont les conducteurs intérieurs ont une géométrie parabolique. L'article \cite{Kopp2006} résume les différences géométries possibles et leurs intérêts. Les cornes paraboliques sont capables de focaliser les pions d'une impulsion données pour n'importe quel angle incident. Une seconde corne permet d'améliorer la focalisation, en récupérant les particules trop ou trop peu focalisées par la première cornes. La \autoref{fig::horns} schématise le principe d'une corne parabolique et de deux cornes à la suite.
            Dans \gls{dune}, les 50 derniers centimètres de la cible sont dans la première corne magnétique (voir \autoref{fig::horn1}), longue de $\SI{3.36}{\meter}$ et large de $\SI{16.5}{\centi\meter}$. Elle sera parcourue par un courant de $\SI{230}{\kilo\ampere}$. L'entrée de la seconde corne est située à $\SI{3.24}{\meter}$ de la sortie de la première corne. Cette seconde corne, parcourue par le même courant, est longue de  $\SI{3.63}{\meter}$ et large de $\SI{39.5}{\centi\meter}$. Chacune des deux cornes est constituée de deux conducteurs, intérieurs et extérieur, faits en Aluminium Al~6061-T6. Les conducteurs extérieurs sont cylindriques, tandis que les conducteurs intérieurs sont fait de deux paraboles communiquant par une petite ouverture de $\SI{9}{\milli\meter}$ dans la première corne et de $\SI{39}{\milli\meter}$ dans la deuxième. Les cornes magnétiques ont été imaginées par S.Van Der Meer en 1961\cite{VanDerMeer1961}.
            %Détailler windows?
            
            \paragraph{Le tuyau de désintégration:} $\SI{17}{\meter}$ après la sortie de la seconde corne se trouve l'entrée du tuyau de désintégration d'un diamètre de $\SI{4}{\meter}$, long de $\SI{194}{\meter}$ et rempli d'hélium, où les mésons pourront se désintégrer et produire les neutrinos. Ce tuyau est pointé vers le bas, avec une pente de 10~\%. La \autoref{fig::decay_pipe} montre une vue en coupe de face de ce tuyau. 
            
            \paragraph{L'absorbeur:} A la fin du tuyau de désintégration se trouve la cavité contenant l'absorbeur, à $\SI{30}{\meter}$ sous la surface. C'est un assemblage d'aluminium, d'acier et de béton, où les particules chargées restantes seront arrêtées. A cet endroit se trouve aussi une alcôve à muons dont le but est de détecter les muons produits avec les neutrinos lors des désintégrations des mésons. En effet, ces muons nous donnent une information sur la direction, l'intensité, le flux et la largeur du faisceau, qu'il sera important de surveiller durant les opérations.
            
            \paragraph{Le flux:} Les neutrinos issues de ce faisceau auront des énergies comprises entre $\SI{0.5}{\giga\electronvolt}$ et $\SI{5}{\giga\electronvolt}$. Au niveau du détecteur lointain, $\SI{1300}{\kilo\meter}$ plus loin, les pics de probabilité d'oscillation correspondront à des énergies de $\SI{0.8}{\giga\electronvolt}$ et $\SI{2.4}{\giga\electronvolt}$, qui sont donc comprises dans le spectre du faisceau de neutrinos. Les flux attendues avec et sans oscillations au niveau du détecteur lointains sont présenté en \autoref{fig::nu_flux}.
            
            \subsubsection{Le détecteur proche}\label{sec::near_detector}
            Après l'absorbeur, le faisceau est constitué uniquement de neutrinos, et va continuer en ligne droite jusqu'au détecteur proche, situé à $\SI{300}{\meter}$ de là et $\SI{60}{\meter}$ sous la surface. Son rôle est de mesurer en détail les caractéristiques du faisceau de neutrinos, ainsi que les sections efficaces et les produits des interactions de neutrinos. Ceci requière une très bonne résolution. Le détecteur proche sera un trajectographe à grain fin, consistant en un module central de trajectographe à tube entouré de calorimètres électromagnétique, le tout dans un aimant de $\SI{0.4}{\tesla}$ (voir \autoref{fig::ND}). Des identificateur de muon entoureront le tout. Les caractéristiques et performances attendues du détecteur proche sont résumé dans le tableau \ref{tab::ND-perf}.
            
            \subsubsection{Le détecteur lointain}\label{sec::far_detector}
            
            La technologie du détecteur lointain est décrit en détail en \autoref{sec::lartpc}.
            
            Après avoir parcourue \SI{1300}{\kilo\meter} à travers la croûte terrestre, les neutrinos atteindrons le détecteur lointain, constitué de 4 modules contenant chacun \SI{10}{\kilo\tonne} d'argon liquide. Ce seront des chambres a projection temporelle, offrant une grande précision en terme de reconstruction de vertex d'interaction ainsi qu'une très bonne résolution en énergie, tout en étant assez dense pour permettre de détecter des neutrinos. Un de ces module pourra utiliser la technologie à double phase d'argon, encore en étude et décrite en \autoref{sec::dlartpc}, afin d'amplifier les charges à détecter.
            
            Donner des chiffres! L/E, flux attendu, graphs...
            comment les objectifs vont être atteints (théorie, flux...)
    
    \section{\texorpdfstring{protoDU$\nu$E}{protoDUNE} : la technologie de Chambre a projection temporelle a argon liquide}\label{sec::lartpc}
    
        \subsection{Introduction}
            Faire le lien avec la section précédente, objectifs, localité du projet, labo impliqués
    
        \subsection{Les Chambres à Projection Temporelle}
            commencer par un petit récap des technologie précédente (chambre à bulle, brouillard...) et lister les avantages de la TPC. Exemples d'expérience utilisant des TPC à gaz (ALICE), finir sur les TPC à liquides.
        
        \subsection{Pourquoi l'argon liquide?}
            Dense, peu cher, \\
            proto single phase = qq resultats
        
        \subsection{La technologie à double phase d'argon}
            \subsubsection{Avalanche de Townsend}
                chambre à file, toussa toussa.
            \subsubsection{Avantages et inconvénients par rapport à la technologie simple phase}\label{sec::townsend_avalanche}
            \subsubsection{État de l'art}
            \subsubsection{Les points peu étudiés}
                UV réémission
        
        \subsection{Les TPC de \texorpdfstring{protoDU$\nu$E}{protoDUNE}}
            \subsubsection{le 311}
                pas plus de quelques paragraphe, il sera décrit plus en détail au chapitre 4
            \subsubsection{le 666}
                un peu plus que sur le 311