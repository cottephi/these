\chapter{Contexte théorique et expérimental}
    \chapterprecishere{
        ``Potentielle citation sans aucun rapport avec le sujet"\par\raggedleft--- \textup{Personne inconnue}, contexte à déterminer
    }
    
    Le but de ce chapitre est de couvrir en quelques pages l'histoire des neutrinos, d'un point de vue théorique comme expérimental, de leur découverte jusqu'aux questions encore non résolues.
    
    \section{De la nécessité théorique à la découverte du neutrino}
    
        \subsection{Le spectre de désintégration \texorpdfstring{$\beta$}{b} : besoin d'une nouvelle particule}
    
        \subsection{Premières observations directes du neutrino}
    
        \subsection{Les 3 familles de neutrinos}
    
    \section{Le paradigme des oscillations des neutrinos}
    
        \subsection{Premières théories et observations}
            Pontecorvo, oscillations à deux saveurs, Homestake experiment, déficit du flux de neutrinos solaires
        
        \subsection{Les oscillations à trois saveurs et les limites des approximations du modèle actuel}
            matrice PMNS, SNO, superK, prix nobel
            approx : ondes planes, unitarité de U, U est 3x3, séparation des paquets d'onde (supernovae)
        
        \subsection{Les différentes sources de neutrinos et ce qu'elles nous permettent de mesurer}
    
    %\section{La violation CP et l'asymétrie matière / anti-matière}