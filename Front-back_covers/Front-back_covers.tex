%https://www.universite-paris-saclay.fr/fr/la-these-contenu-langue-de-redaction-droit-dauteur-confidentialite-format-et-page-de-couverture

\frontmatter
\usetikzlibrary{calc}
\thispagestyle{empty}

\chapter*{}
%\addcontentsline{toc}{chapter}{Page de garde}%
\thispagestyle{empty}



%% Positionner le cadre dans la page.
%% Modifier yshift modifie la position des bords haut et bas du cadre.
%% Modifier xshift modifie la position des bords gauche et droit du cadre.
%% Il faut toujours les modifier deux par deux (ceux qui ont la même valeur ensemble).
\begin{tikzpicture}[remember picture,overlay,color=pheniics_purple]
	\draw[very thick]
		([yshift=-120pt,xshift=30pt]current page.north west)--     
		([yshift=-120pt,xshift=-30pt]current page.north east)--    
		([yshift=50pt,xshift=-30pt]current page.south east)--      
		([yshift=50pt,xshift=30pt]current page.south west)--cycle; 
\end{tikzpicture}





%% Logos en haut de la page
\begin{textblock}{13.8}(1,0.5)
	\includegraphics[width=\textwidth]{Logo_ALL.png} %% Logo de Paris Saclay
	\label{Logo Paris Saclay}
\end{textblock}

%\begin{textblock}{12.8}(14.5,2)
%	\includegraphics[height=2.4cm]{Title/Logos/UPS.png} %% Logo de votre établissement
%	\label{Logo Etablissement}
%\end{textblock}


%% Position du NNT
\begin{textblock}{12.8}(0.7,2.45)
	NNT : \textcolor{red}{[NNT number]}
\end{textblock}



\begin{textblock}{12.8}(1.6,3.5)
  %% Texte
  \begin{center}
    \textcolor{pheniics_purple}{ %% Couleur violette du premier paragraphe
      \LARGE\textsc{Thèse de doctorat\\ de l'Université Paris-Saclay} \\
      \LARGE{\textsc{préparée à l'Université Paris-Sud}} \\ \bigskip
  	  \color{black} %% Couleur noir du reste du texte
	  \vfill \vfill
	  \Large{CEA/Irfu/DPhP}
	  \vfill \vfill
      \Large\textsc{École doctorale n$^{\circ}576$}\\ %% Numéro ED
      \Large{Particules, Hadrons, Énergie, Noyau, Instrumentation, Imagerie, Cosmos et Simulation (PHENIICS)} \\
	  \Large{Spécialité de doctorat: \textcolor{red}{Physique des neutrinos} } %% Spécialité
      \vfill  
   	  \Large{par}
   	  \vfill
   	  \LARGE{\textbf{\textsc{Philippe Cotte}}} %% Nom du docteur
      \vfill
      \Large{Le projet WA105 : un prototype de Chambre à Projection Temporelle à Argon Liquide Diphasique utilisant des détecteurs LEMs} %% Titre de la thèse
      \vfill
    }
  \end{center}

  \small{
%% Jury
    \begin{flushleft}
    Thèse présentée et soutenue à Gif-Sur-Yvette le \textcolor{red}{17 septembre 2019}. \\
    \bigskip
    Composition du jury :
    \end{flushleft}
%% Members of the jury
%% If needed, one can add jurymemberG or remove one jury member.

    \begin{center}
      \begin{tabular}{llll}

	  % Président du jury, Rapporteur, Examinateur
	
	    Pr.		& A. Tonazzo,			& Université Paris-Diderot, Laboratoire
APC,		& Présidente du Jury	\\
  	    Dr		& A. Meregaglia,			& Université de Bordeaux, CENBG,		& Rapporteur			\\	
  	    Dr		& I. Gil Botella,			& CIEMAT, Madrid,		& Rapporteur			\\
  	    Dr.		& E. Baussan,			& Université de Strasbourg, IPHC,		& Examinateur		\\
  	    Dr.		& P. Schune,			& CEA-Paris Saclay,		& Examinateur		\\
	    Dr.		& E. Mazzucato,			& CEA-Paris Saclay,		& Directeur de thèse	\\  
   
      \end{tabular}    
    \end{center}
  }

\end{textblock}

%% Logos en bas de la page
\begin{textblock}{3}(0.63,13.2)
	\includegraphics[width=\textwidth]{logo_CEA.png} %% Logo de Paris Saclay
	\label{Logo CEA}
\end{textblock}

%%%%%%%%%%%%%%%%%%%%%%%%%%%
%%%%%%%%%%% old %%%%%%%%%%%%%%
%%%%%%%%%%%%%%%%%%%%%%%%%%%

%
%\frontmatter
%\usetikzlibrary{calc}
%\thispagestyle{empty}


%%%% the purple border line %%%
%\begin{tikzpicture}[remember picture, overlay]
%    \draw[line width=1.2 pt, pheniics_purple] 
%    ($(current page.south west)+(1 cm,+1. cm)$) 
%    -- ($(current page.north west)+(1 cm,-1. cm)$) 
%    -- ($(current page.north east)+(-1 cm,-1. cm)$) 
%    -- ($(current page.south east)+(-1 cm,1. cm)$)
%    -- ($(current page.south west)+(1 cm,1. cm)$);
%\end{tikzpicture}
%
%{\begin{center}
%	\vspace{-3.5cm}
%	%%% logo %%%
%	\includegraphics[width=14cm]{Logo_ALL.png}\\
%	\vspace{1cm}
%	
%	%%% university title %%%
%	\textcolor{violet!80!red!80!black}{{{\uppercase{\Large Thèse de Doctorat de L'Université Paris-Saclay Préparée à l'Université Paris-Sud}}}}\\
%	\vspace{1cm}
%	%%% doctoral school title %%%
%	ÉCOLE DOCTORALE N$^{\circ}$576\\
%	Particules Hadrons Énergie et Noyau : Instrumentation, Image, Cosmos et Simulation (PHENIICS)\\
%	Spécialité de doctorat : Physique des neutrinos expérimentale\par
%	\vspace{1.5cm}
%	%%% name %%%
% 	Par\par  \large \textbf{M. Philippe Cotte} \par
%	\vspace{1cm}
%	%%% thesis title %%%
%	\Large \textsc{\textcolor{SchoolColor}{
%	\textbf{Le projet WA105 : un prototype de Chambre à Projection Temporelle à Argon Liquide Diphasique utilisant des détecteurs LEMs}}}\par
%\end{center}
%
%\vspace{2cm}
%\hspace{-1cm}{\textit{Thèse présentée et soutenue à Orsay, le 1 septembre 2016} \par}
%\vspace{1cm}
%\hspace{-1cm}{  Composition de jury: \par}
%\hspace{-1cm}{  M. Hubert Farnsworth, \textit{Professeur, Université de Lefutur,} Rapporteur \par}
%\hspace{-1cm}{  M. Malcolm Reynolds, \textit{Professeur, Université de Serenity,} Rapporteur \par}
%\hspace{-1cm}{  M. Derrial Book, \textit{Professeur, Université de l'Aliance,} Examinateur \par}
%\hspace{-1cm}{  Mme Turanga Leela, \textit{Chargée de Recherche, TU Arcturan,} Examinatrice \par}
%\hspace{-1cm}{  Mme Amy Wong, \textit{Directrice de recherche, Planet Express,} Directrice de thèse \par}
%\hspace{-1cm}{  M. John Zoidberg, \textit{Chargé de Recherche, IPN Decapod,} Co-directeur de thèse \par}
%}
%
%
%% ~~~~~~~~~~~~~~~~~~~~~~~~~~~~~~~~~~~~~~~~
%% ~~~~~~~~~~~~~~~~~~~~~~~~~~~~~~~~~~~~~~~~
%
%%%% a lifehack to adgust the font size and spacing %%%
%\makeatletter
%\newcommand*\mysize{%
%  \@setfontsize\mysize{9.5}{9.0}%
%}
%\makeatother
%
%\newpage
%\thispagestyle{empty}
%\begin{tikzpicture}[remember picture, overlay] 
%\end{tikzpicture}
%     
%\newpage
%\thispagestyle{empty}
%\begin{tikzpicture}[remember picture, overlay]
%	%%% the University+ED logo %%%
%    \node [anchor=north west, shift={(1.2 cm,-0.2cm)}] at (current page.north west) {\includegraphics[width=7.5cm]{pheniics.png}};
%     \mysize 
%    \node [anchor=north, yshift=-2.1 cm, text width=18cm, inner sep=.3cm] (resume) at (current page.north) {
%    \begin{minipage}{\linewidth}
%    %%% title %%%
%\justify{     {\textbf{Titre:}} Un titre long et beau qui prend probablement plus d'une ligne\\
%	%%% key words %%%
%     			  {\textbf{Mots clés:}} \textit{astrologie, exo-psychologie, arts sombres, Voyage spatial}\\       		
%     			  {\textbf{Résumé:}}  \lipsum[1-3] %%% replace by the text of the abstract in French %%%
%}
%    \end{minipage}
%    };
%
%    
%    %%% draw a purple frame around each abstract %%%
%    \draw[line width=1 pt, violet!80!red] (resume.south west) -- (resume.north west) -- (resume.north east) -- (resume.south east) -- (resume.south west);
%    
%    %%% footnote %%%
%    \node [anchor=south west, violet!80!red, shift={(1.2 cm,0.5cm)}, inner sep=0.2pt] at (current page.south west) {
%    \begin{minipage}{12cm}
%    {\textbf{Université Paris-Saclay}} \\
%    Espace Technologique / Immeuble Discovery \\
%    Route de l'Orme aux Merisiers RD 128 / 91190 Saint-Aubin, France 
%    \end{minipage}
%    };
%    
%    %%% the "e" image at the bottom %%%
%    \node [anchor=south east, violet!80!red!80!black, shift={(-1.5 cm,0.5cm)}, inner sep=0pt] at (current page.south east) {\includegraphics[width=1.6 cm]{e.png}};
%    
%\end{tikzpicture}
%
%\newpage
%\thispagestyle{empty}
%\begin{tikzpicture}[remember picture, overlay] 
%\end{tikzpicture}
%
%\newpage
%\thispagestyle{empty}
%\begin{tikzpicture}[remember picture, overlay]
%	%%% the University+ED logo %%%
%    \node [anchor=north west, shift={(1.2 cm,-0.2cm)}] at (current page.north west) {\includegraphics[width=7.5cm]{pheniics.png}};
%     \mysize 
%    \node [anchor=north, yshift=-2.1 cm, text width=18cm, inner sep=.3cm] (resume) at (current page.north) {
%    \begin{minipage}{\linewidth}
%    %%% title %%%
%    %%% title %%%
%\justify{     {\textbf{Title:}} A long and beautiful title that probably takes more than one line\\
%	%%% key words %%%
%     			  {\textbf{Key words:}} \textit{astrology, exo-psychology, dark arts, space travel} \\
%    			  {\textbf{Abstract:}} \lipsum[1-3]  %%% replace by the text of the abstract in English %%%
%}
%    \end{minipage}
%    };
%
%    
%    %%% draw a purple frame around each abstract %%%
%    \draw[line width=1 pt, violet!80!red] (resume.south west) -- (resume.north west) -- (resume.north east) -- (resume.south east) -- (resume.south west);
%    
%    %%% footnote %%%
%    \node [anchor=south west, violet!80!red, shift={(1.2 cm,0.5cm)}, inner sep=0.2pt] at (current page.south west) {
%    \begin{minipage}{12cm}
%    {\textbf{Université Paris-Saclay}} \\
%    Espace Technologique / Immeuble Discovery \\
%    Route de l'Orme aux Merisiers RD 128 / 91190 Saint-Aubin, France 
%    \end{minipage}
%    };
%    
%    %%% the "e" image at the bottom %%%
%    \node [anchor=south east, violet!80!red!80!black, shift={(-1.5 cm,0.5cm)}, inner sep=0pt] at (current page.south east) {\includegraphics[width=1.6 cm]{e.png}};
%    
%\end{tikzpicture}
