\section{Introduction}
\subsection{Contexte}
%\begin{itemize}
	%\item  Domaine : Physique des neutrinos : peuvent changer de saveur -> masse -> BSM -> WAAAAH!
	%\item Problème : Phase de violation CP, hiérarchie de masse et autres phénomènes rares nécessitent une détermination précises des probabilité de transition de saveurs vs L/E. Or neutrino interagissent peu (libre parcours moyen ~année lumière).
	%\item  Conséquences : Il faut des détecteur gros et dense (équivalent en eau : plrs dixaines de kt) pour augmenter les chances d'interaction et très précis (res énergie < 3\%) pour déterminer proba d'oscillation vs L/E.
%\end{itemize}
Les résultats des expériences Super-Kamiokande (1997) et SNO (2005) ont montré que les neutrinos peuvent changer de saveurs entre leur création et leur détection, ce qui a value le prix Nobel de physique en 2015 à Takaaki Kajita (SuperK) et Arthur B. McDonald.  Ce phénomène de changement de saveur n'est possible que si les neutrinos ont une masse non nulle. Le modèle standard de la physique des particules ne prend pas en compte ces masses, leurs valeurs étant trop faible pour être mesurables. Ce dernier doit donc être étendu afin d'inclure les masses des neutrinos. Il existe plusieurs théories capables de le faire, mais ces différentes théories dépendent de la hiérarchie des masses des neutrinos, encore inconnue. De plus, le modèle théorique prédisant le changement de saveur des neutrinos inclue une phase pouvant violer la symétrie Charge-Parité. Si cette phase est différente de 0 et $\pi$, la matière et l'antimatière ne se comportent pas tout à fait de la même manière, ce qui peut expliquer pourquoi l'antimatière a entièrement disparue de l'univers. La hiérarchie de masse et la phase de violation CP peuvent être déterminées dans une expérience d'oscillation de neutrinos à longue ligne de base d'accélérateur en mesurant la probabilité de transition de saveur des neutrinos en fonction de du rapport de la distance $L$ parcourue par ces derniers et de leur énergie $E$. Cette probabilité de transition oscille avec $L/E$ via une somme de sinus et cosinus, le terme dominant n'étant sensible ni à la hiérarchie de masse ni à la phase de violation CP. Aussi une très bonne résolution en énergie et en espace est nécessaire pour être sensible aux différentes modulation de la probabilité de transition. De plus, les neutrinos interagissent très peu avec la matière : leur libre parcours moyen est de l'ordre de l'année lumière. Pour pouvoir avoir une statistique suffisante à la détermination de la phase de violation CP et de la hiérarchie de masse, il faut disposer d'une source avec un grand flux de neutrino et d'un détecteur le plus gros et le plus dense possible. L'expérience \gls{dune}, prévu pour 2030 aux états unis, se propose de mesurer la hiérarchie de masse et la phase de violation CP, avec d'autres phénomènes rares comme la désintégration du proton. Elle détectera des neutrinos d'un faisceau produit par le Fermilab, près de Chicago, dans 4 modules de détection de \SI{10}{\kilo\tonne} chacun utilisant la technologie de \gls{lartpc}, situés à Sanford dans le Dakota du Nord.

\subsection{Etudes : 10 lignes par auteurs}
auteur A | a fait : | a permit : | limites :\\
Pour DUNE :\\
\begin{itemize}
	\item SNO | changement de saveurs des $\nu$ solaires et mesure de 2 param de la matrice de mélange | pas d'info sur les autres param 
	\item SK | changements de saveurs de $\nu$ atm et mesure de 2 param de la matrice de mélange | pas d'info sur le reste
	\item Daya Bay | mesure dernier angle de mélange -> Violation CP possible | pas d'info sur $\delta_{CP}$ et MH
	\item T2K | $\delta_{CP}$ non nulle à $2\sigma$
\end{itemize}
\gls{dune} se propose de mesurer $\delta_{CP}$ et MH avec TPC à argon liquide (HK fera de même avec water Cerenkov).

Pour protoDUNE : 
\begin{itemize}
	\item Rubbia | propose un nouveau type de détecteur | ne fait que proposer
	\item ICARUS | détecter $\nu$ avec \gls{lartpc} ok sur le principe, a permis de mesurer des grandeurs utiles comme recombinaison à l'ionisation et temps de vie des électrons | trop petit et pas assez précis. 
\end{itemize}
Deux choix : pousser la techno \gls{lartpc} (safe car ICARUS était presque assez précis, donc il suffit de changer d'échelle) ou essayer la techno avec ampli de charge \gls{dlartpc} (moins safe mais permettrait d'accéder à des énergies plus basse et d'être plus précis). ProtoDUNE fait les deux et \gls{wa105} test le \gls{dlartpc}.

Pour WA195 :
\begin{itemize}
	\item A. Rubbia | propose \gls{dlartpc} | ne fait que proposer
	\item Townsend | a permit montrer que l'ampli de charge dans GAr est possible et a étudier gain vs ampli pour des modèles simples | uniquement GAr, pas de phase liquide.
	\item 3L | Avec phase liquide : a permit d'identifier les géométries d'amplificateurs et d'anode les plus efficaces (haut gain, bonne stabilité dans le temps) et a pu vérifier que le comportement gain vs ampli correspond au modèle de Townsend | petit format : pas de contraintes mécaniques, et drift max de 10 cm -> peu de pertes. Un seul module LEM-Anode : pas d'interactions entre plusieurs modules.
	\item 250L : je c pa
	\item gamelle | test des modules prévues pour DUNE dans l'argon gazeux | par d'argon liquide, petite taille également
\end{itemize}
D'où la nécessité de tester DLArTPC à l'échelle du kt

\subsection{Bilan : 10 lignes}
Un rapide bilan montre que la mesure précise du phénomène d'oscillation des neutrinos fournira un élément de réponse à la question "pourquoi y a-t-il quelque chose plutôt que rien" et de résoudre la question de la hiérarchie de masse des neutrinos. Ces mesures peuvent se faire avec avec la technologie \gls{lartpc} proposé par C. Rubbia en 1977.  Les études menées par ICARUS ont montré la possibilité de reconstruire en 3D les interactions de neutrino dans l'argon liquide d'une \gls{lartpc}, avec une très bonne précision à la fois en espace et en énergie.  La proposition d'amélioration de cette technologie, qui consiste à amplifier le signal dans une phase d'argon gazeux, permet de diminuer à la fois le rapport signal sur bruit et le seuil de détection sur des distances de dérive allant jusqu'à $\SI{10}{\meter}$. Elle a été testée à petite échelle par un prototype de 3L et un autre de 250L. Cependant, lblablabla

%Une des expériences qui se propose de faire ceci est \gls{dune}, qui détectera les interactions des neutrinos initialement muoniques d'un faisceau issue du Fermilab. L'énergie d'un de ces neutrinos peut être comprise entre 0 et \SI{5}{\giga\electronvolt} ce qui couvre deux maxima de probabilité de changement de saveur, sachant que la distance parcourue est de \SI{1300}{\kilo\meter}. La mesure précise de la probabilité de changement de saveur en fonction de $L/E$ se fera avec 4 modules de $\SI{10}{\kilo\tonne}$ rempli d'argon liquide utilisant la technologie \gls{lartpc}. Un de ces modules sera la version double phase de cette technologie, permettant de détecter des énergies plus basses et d'avoir une meilleure résolution spatiale. Les études actuelles sur cette technologie ont montré que le principe d'amplification fonctionne, mais avait une taille deux ordres de grandeur en dessous de ce qui \gls{dune} a prévu. 

\subsection{D'où mon sujet}
Le but du projet WA105 est d'étudier les capacités d'une \gls{dlartpc} à l'échelle de la tonne à remplir les objectifs de \gls{dune} en exploitant les données de rayons cosmiques détectés par un prototype de $3\times1\times\SI{1}{\meter^3}$ puis par un module de détection de \gls{dune} de $6\times6\times\SI{6}{\meter^3}$. En effet, une étape de prototypage en plus est nécessaire afin de passer de \SI{3}{\liter} à \SI{10}{\kilo\tonne}, afin d'anticiper et résoudre les problèmes liés aux grands volumes et grandes surfaces de détection dans une \gls{dlartpc}. Le comportement du gain observé dans le prototype de \SI{3}{\liter} peut-il être reproduit par une plus grande surface du plan de lecture de charge? Ce comportement est-il compris et peut-il être utilisé pour déduire la charge avant amplification? Cette surface permet-elle d'atteindre les mêmes gains que le prototype  de \SI{3}{\liter}? Les impuretés dans un grand volume sont-elles un frein à la reconstruction des traces les plus loin du plan de lecture du charge? Le comportement du gain peut-il être extrapolé aux dimensions de \gls{dune}? La résolution en énergie est-elle suffisante pour atteindre les objectifs de \gls{dune}? L'étude présentée ici s'intéresse au résultats du prototype de $3\times1\times\SI{1}{\meter^3}$ qui a été assemblé au \gls{cern}, à Genève, et a détecté des cosmiques entre Juin et Octobre 2017, et prépare l'exploitation du démonstrateur de $6\times6\times\SI{6}{\meter^3}$, en cours d'assemblage au \gls{cern} et qui devrait prendre des données à partir de l'été 2019.

\subsection{La démarche : 10 lignes par chapitre}
L'étude est faite en deux temps. Une partie est dédiée aux tests du plan de lecture de charge du démonstrateur $6\times6\times\SI{6}{\meter^3}$ puis s'intéresse aux résultats du $3\times1\times\SI{1}{\meter^3}$.

L'objectif du \autoref{chap::666} est de prédire les performances et les limitations du design des plans de lecture de charge prévue pour \gls{dune}. Dans un premier temps, une étude de l'impact des zones mortes des amplificateurs sur la charge collectée a été réalisée. L'idée a été de simuler la dérive à travers le plan de lecture de charge des électrons d'un signal produit proche de ces zones et de regarder leur point d'arriver, afin de prédire la proportion de charge vue par les canaux de lecture situés au niveau de ces zones. Dans un second temps, les variations d'épaisseurs de 72 amplificateurs de $50\times\SI{50}{\centi\meter\squared}$ ont été mesuré sur une table optique. En effet, l'épaisseur d'un amplificateur influence son gain de manière exponentiel. Les inévitables fluctuations de cette épaisseur constituent une incertitude irréductible sur le gain, qu'il convient de connaître.  Dans un troisième temps, les résultats de mesures de gain des mêmes amplificateurs dans une gamelle haute pression remplie d'argon gazeux sont étudiés pour prédire le comportement  du prototype $3\times1\times\SI{1}{\meter^3}$ et du démonstrateur $6\times6\times\SI{6}{\meter^3}$.  Dans la même gamelle ont été effectués des tests de tenu en tension des amplificateur afin de prédire le gain maximal atteignable. Ces même tests ont été répétés dans une boîte cryogénique pour les plans de lecture de charge complets du démonstrateur $6\times6\times\SI{6}{\meter^3}$, avec leurs 36 amplificateurs et anodes de lecture ainsi que leur grille d'extraction.

L'objectif du \autoref{chap::311} est d'étudier le comportement du gain dans le prototype de $3\times1\times\SI{1}{\meter^3}$ afin de prédire le comportement du $6\times6\times\SI{6}{\meter^3}$ et, plus tard, de \gls{dune}. Une première étape consiste à produire un facteur de correction dépendant de la densité de l'argon, variant légèrement au cours des prises de données. Cette dernière, disponible à la secondes près, influence en effet grandement le gain. Pour pouvoir étudier le gain en fonction du champ d'amplification, il faut donc s'affranchir de cette dépendance. Une seconde étape a été de déterminer la pureté de l'argon en regardant la variation de la charge collectée avec la distance au plan de collection de charge, afin de déterminer si cette impureté est acceptable pour les besoin de \gls{dune}, et également de pouvoir, par la suite, s'affranchir des effets d'impuretés dans l'étude du gain. La troisième étape a été de prédire le comportement de la charge collectée en fonction des champs d'extraction et d'induction. En effet, une certaine quantité de charge peut être perdue avant et après amplification, quantité qui dépend des lignes de champs à travers le plan de détection de charge. Pour cela, l'idée a été de simuler les pertes d'électrons dans le plan de lecture de charges pour plusieurs configurations de champs, afin de pouvoir corriger la charge mesurée pour ces pertes et s'affranchir de l'effet des champs d'extraction et d'induction dans l'analyse du comportement du gain. Une comparaison aux données du $3\times1\times\SI{1}{\meter^3}$ du comportement de la charge en fonction du champ d'extraction a été faite à un champ d'amplification de  \SI{28}{\kilo\volt\per\centi\meter} et un champ d'induction de ***mettre la valeur***. Nous n'avons pas pu effectuer cette comparaison à un autres champs d'amplification, ni en fonction du champ d'induction, à cause du manque de données causé $3\times1\times\SI{1}{\meter^3}$. La dernière étape a été d'étudier le comportement du gain en fonction du champ d'amplification afin d'estimer le gain maximal atteignable dans le $6\times6\times\SI{6}{\meter^3}$ ainsi que son comportement.

%L'objectif du \autoref{chap::collection_probability} est de rendre possible l'étude du comportement du gain en fonction du champ d'amplification dans le prototype $3\times1\times\SI{1}{\meter^3}$. En effets, comprendre ce comportement permettra au démonstrateur $6\times6\times\SI{6}{\meter^3}$ de prédire la capacité d'une \gls{dlartpc} à atteindre les objectif de \gls{dune}. La charge collectée par le \gls{crp} est influencée par les champs d'extraction et d'induction, il faut donc pouvoir décorréler les effets de ces champs sur la charge à celui du champ d'amplification. Pour cela, l'idée a été de simuler les pertes d'électrons dans le plan de lecture de charges pour plusieurs configurations de champs, afin de pouvoir corriger la charge mesurée pour ces pertes et s'affranchir de l'effet des champs d'extraction et d'induction dans l'analyse du comportement du gain. Pour ce faire, le champ électrique à travers le plan de lecture de charge a été simulé. Une approximation géométrique de la grille de lecture de charge a été faite afin de permettre des temps de calcul raisonnables avec le logiciel utilisé, ANSYS, qui est capable de produire des cartes de champs pouvant être utilisées par le logiciel Garfield. Ce dernier est capable de simuler la dérive et l'amplification des électrons dans un gaz en présence d'une carte de champ électrique. Dans un second temps, nous avons vérifié que la simulation de l'influence du champ d'extraction sur la charge collectée est identique à celle mesurée par le $3\times1\times\SI{1}{\meter^3}$. Cette comparaison a été faite à un champ d'amplification fixé à \SI{28}{\kilo\volt\per\centi\meter}. Nous n'avons pas pu effectuer cette comparaison à un autres champs d'amplification, ni en fonction du champ d'induction, à cause du manque de données.

%Le chapitre 2 a pour objectif de comprendre le comportement du gain en fonction du champ d'amplification. Pour ce faire ...

%- Elle a consisté a : objectif(chap 1) -> objectif(chap 2) etc... SANS PASSER PAR lA CASE RESULTAT. On peut utiliser la phrase "l'étape suivante a consisté à obj(1) pour obj(2)".
%  -  Pour un obj : L'obj(i) a été de gnagnagna. POUR CELA
    %L'idée est de A
    %Ensuite l'idée est de B
    %\label{key}...