\chapter{Contexte théorique et expérimental}
    \chapterprecishere{
        ``Potentielle citation sans aucun rapport avec le sujet"\par\raggedleft--- \textup{Personne inconnue}, contexte à déterminer
    }
    
    Le but de ce chapitre est de couvrir en quelques pages l'histoire des neutrinos, d'un point de vue théorique comme expérimental, de leur découverte jusqu'aux questions encore non résolues.
    
    \section{De la nécessité théorique à la découverte du neutrino}
    
        \subsection{Le spectre de désintégration \texorpdfstring{$\beta$}{b} : besoin d'une nouvelle particule}
    
        \subsection{Premières observations directes du neutrino}
    
        \subsection{Les 3 familles de neutrinos}
    
    \section{Le paradigme des oscillations des neutrinos}
    
        \subsection{Genèse de la théorie}
        
            On désigne habituellement un neutrino par sa saveur : neutrinos muonique($\nu_{\mu}$), électronique($\nu_e$) ou tauique$\nu_{\tau}$. Un neutrinos est dans l'un de ces 3 "états de saveur" si il est produit par le lepton qui y correspond, ou si il produit ce lepton en interagissant avec son environnement. Rien, à priori, n'oblige le lepton qui créé le neutrino à être de même saveur que le lepton créé par le neutrino après interaction. Le premier à avoir soulevé ceci est Bruno Pontecorvo, même si il ne l'a pas fait en ces termes. En effet, au moment de la publication de ses deux premiers articles\cite{Pontecorvo:1957cp,Pontecorvo:1957qd} sur le sujet à la fin des années 60, il n'était pas connu qu'il y avait plusieurs saveurs de neutrino. B.~Pontecorvo parlait de possible transition entre neutrino et antineutrino du fait que le neutrino soit neutre, inspiré par les travaux de Gell-Mann et Païs\cite{Gell-Mann1955} sur la conversion du $\bar{K^0}$ en  $K^0$. Dans son article suivant en 1968\cite{Pontercorvo1968}, tout en gardant la possibilité de conversion des neutrinos vers les antineutrinos, il introduit la possibilité d'une conversion du neutrino électronique vers le neutrino muonique, découvert en 1962\cite{Danby1962}. Il prédira également deux résultats importants :
            \begin{itemize}
                \item Si les masses des neutrinos ne sont pas nulles et que la charge leptonique n'est pas être conservée, les neutrinos peuvent changer de saveur
                \item Un déficit de neutrinos en provenance du soleil d'un facteur 2 environ par rapport au prédiction du modèle solaire standard est attendu dans ce cas
            \end{itemize}
            La première prédiction implique de la physique au delà du modèle standard de la physique des particules, puisque ce dernier suppose que les masses des neutrinos sont nulles. Mais ceci n'est pas imposé par la théorie : il a été expérimentalement observé dans toutes les expériences à ce jour que les neutrinos se comportent, si ce n'est comme des particules de masses nulles, comme des particules de masse suffisamment faible pour être négligée. 
            La dernière prédiction fut vérifiée en 1970 par la Brookhaven Solar Neutrino Experiment\cite{Bahcall1976}, qui trouva un déficit compris entre 2 et 3.
            
            Le changement de saveur $e\rightleftharpoons\mu$ avait été envisagé également entre 1962 et 1963 par deux groupes de physiciens, Katayama, Matumoto, Tanaka et Yamada\cite{Nakagawa1963} puis que Ziro Maki, Masami Nakagawa and Shoichi Sakata\cite{Maki1962}. Ces quatre derniers donneront leurs noms, avec Pontecorvo, à la célèbre matrice \gls{pmns} décrite plus loin. Leur point de départ était différent de celui de Pontecorvo, puisqu'ils visaient à créer une théorie unifiant les leptons et les hadrons. Ils sont également arrivé à la conclusion qu'une oscillation des neutrinos implique que ces derniers doivent avoir des masses non nulles.
    
        \subsection{La matrice PMNS}
            
        
        \subsection{Les oscillations à trois saveurs et les limites des approximations du modèle actuel}\label{sec::oscillations}
            \cite{Nunokawa2007}
            matrice PMNS, SNO, superK, prix nobel
            approx : ondes planes, unitarité de U, U est 3x3, séparation des paquets d'onde (supernovae)
        
        \subsection{Les différentes sources de neutrinos et ce qu'elles nous permettent de mesurer}
    
    \section{La violation CP et l'asymétrie matière / anti-matière}\label{sec::CP_violation}
        effets de matiere: \cite{Wolfenstein1978,Mikheyev1986}
        leading order approx \cite{Marciano2006}