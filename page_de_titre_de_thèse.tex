%
% \usepackage[english,frenchb]{babel} % langue principale = french
%
%\usepackage{multicol} % To write summary in two columns-mode
%\usepackage{lipsum}
%\usepackage[absolute]{textpos} % to place elements in the page
%\usepackage[usenames,dvipsnames,svgnames,table]{xcolor} %% To color in violet the text
%
% https://www.universite-paris-saclay.fr/fr/la-these-contenu-langue-de-redaction-droit-dauteur-confidentialite-format-et-page-de-couverture

\chapter*{}
\addcontentsline{toc}{chapter}{Page de garde}%
\thispagestyle{empty}



%% Positionner le cadre dans la page.
%% Modifier yshift modifie la position des bords haut et bas du cadre.
%% Modifier xshift modifie la position des bords gauche et droit du cadre.
%% Il faut toujours les modifier deux par deux (ceux qui ont la même valeur ensemble).
\begin{tikzpicture}[remember picture,overlay,color=blue!20!red!45!black!75!]
	\draw[very thick]
		([yshift=-120pt,xshift=30pt]current page.north west)--     
		([yshift=-120pt,xshift=-30pt]current page.north east)--    
		([yshift=50pt,xshift=-30pt]current page.south east)--      
		([yshift=50pt,xshift=30pt]current page.south west)--cycle; 
\end{tikzpicture}





%% Logos en haut de la page
\begin{textblock}{12.8}(3,2)
	\includegraphics[height=2.4cm]{Title/Logos/UPsac.pdf} %% Logo de Paris Saclay
	\label{Logo Paris Saclay}
\end{textblock}

\begin{textblock}{12.8}(14.5,2)
	\includegraphics[height=2.4cm]{Title/Logos/UPS.png} %% Logo de votre établissement
	\label{Logo Etablissement}
\end{textblock}


%% Position du NNT
\begin{textblock}{12.8}(3,3.8)
	NNT : \comquest{NNT number}
\end{textblock}



\begin{textblock}{12.8}(3.5,4)

%% Texte
\color{blue!20!red!45!black} %% Couleur violette du premier paragraphe
\begin{center}
    \LARGE\textsc{Thèse de doctorat\\ de l'Université Paris-Saclay} \\
    \LARGE{\textsc{préparée à l'Université Paris-Sud}} \\ \bigskip
  	\color{black} %% Couleur noir du reste du texte
	\vfill \vfill
	\Large{CEA/Irfu/DPhP}
	\vfill \vfill
    \Large\textsc{École doctorale n$^{\circ}576$}\\ %% Numéro ED
    \Large{Particules, Hadrons, Énergie, Noyau, Instrumentation, Imagerie, Cosmos et Simulation (PHENIICS)} \\
	\Large{Spécialité de doctorat: \comquest{Spécialité} } %% Spécialité
    \vfill  
   	\Large{par}
   	\vfill
   	\LARGE{\textbf{\textsc{Clotilde Canot}}} %% Nom du docteur
    \vfill
    \Large{Titre de ma thèse} %% Titre de la thèse
    \vfill
\end{center}

\small{
\color{black}
%% Jury
\begin{flushleft}
Thèse présentée et soutenue à \comquest{lieu} le \comquest{date}. \\
\bigskip
Composition du jury :
\end{flushleft}
%% Members of the jury
%% If needed, one can add jurymemberG or remove one jury member.

\begin{center}
  \begin{tabular}{llll}

	% Président du jury, Rapporteur, Examinateur
	
	
	M.		& ABC,			& Université,		& Président du Jury	\\
	Pr		& ABC,			& Université,		& Rapporteur			\\	
	Pr		& ABC,			& Université,		& Rapporteur			\\
	M.		& ABC,			& Université,		& Examinateur		\\
	M.		& ABC,			& CEA-Saclay,		& Directeur de thèse	\\  
   
  \end{tabular}    
\end{center}
}




\end{textblock}






















