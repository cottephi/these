\chapter{L'expérience \texorpdfstring{DU$\nu$E}{DUNE} et son projet de R\&D~: \texorpdfstring{protoDU$\nu$E}{protoDUNE}}
    \chapterprecishere{
        ``Potentielle citation sans aucun rapport avec le sujet"\par\raggedleft--- \textup{Personne inconnue}, contexte à déterminer
    }
    
    Ce chapitre introduit les principales expériences d'oscillation de neutrinos d'accélérateurs à long ligne de base, dont \gls{dune} fait parti, en expliquant leurs points forts et leurs points faibles. Une section y est dédiée à l'expérience \gls{dune} elle même, une autre section est dédié au projet \gls{wa105} faisant parti des expériences de prototypage de \gls{dune}.
        
    \section{Les expériences d'oscillation de neutrinos d'accélérateurs à long ligne de base, actuelles et futures}
    
        Les expériences à longue ligne de base sont caractérisées par le fait de chercher à atteindre un ou plusieurs maxima de probabilité d'oscillation (voir \autoref{sec::oscillations}) en plaçant un détecteur loin de la source et en cherchant à détecter des neutrinos d'énergies correspondantes à ce ou ces maxima. Dans cette section, nous nous intéressons aux expériences utilisant des neutrinos issus d'un flux créé par l'homme dans un accélérateur de proton.
    
        \subsection{Qu'est ce qu'une expérience d'oscillation de neutrinos d'accélérateurs à longue ligne de base?}
            
            Comme il a été discuté en \autoref{sec::oscillations}, la probabilité qu'a un neutrino de changer de saveur oscille avec le ratio $L/E$, où $L$ est la distance entre le point d'origine du neutrino et l'endroit où il est détecté (ligne de base), et $E$ est l'énergie de ce neutrino. Il existe donc des ratios $L/E$ privilégiés pour la détection des changements de saveur, où cette probabilité est localement maximale. Les expériences cherchant à détecter les oscillations des neutrinos cherchent généralement à se placer à une ligne de base tel que $L/E$ couvre un ou plusieurs de ces maxima. Pour ce qui est des expériences détectant des neutrinos solaires ou atmosphériques, la ligne de base ne peut pas être choisie, le but est donc de créer des détecteurs dont la sensibilité en énergie permet de se trouver au niveau de ces maxima, quand le spectre en énergie des neutrinos incidents le permet. Dans les expériences détectant les neutrinos issus de réacteurs nucléaires, c'est le spectre en énergie qui est fixée à quelques \si{\mega\electronvolt}. Pour les expériences détectant des neutrinos issus de faisceaux produits par des accélérateurs, les énergies disponibles sont comprises entre \SI{0.5}{\giga\electronvolt} et \SI{100}{\giga\electronvolt}. Il est donc alors possible d'ajuster à la fois l'énergie et la ligne de base. Un article de 2015\cite{Bass2015} montre que la ligne de base optimale pour déterminer à la fois la phase de violation CP, la hiérarchie de masse et l'octant de l'angle $\theta_{23}$ à $5\sigma$ chacun doit être comprise entre 1000 et \SI{1500}{\kilo\meter}, ce qui a motivé le choix des \SI{1300}{\kilo\meter} de \gls{dune}.
            
            Les expériences d'oscillation de neutrinos d'accélérateurs à longues lignes de base cherchent à mesurer la probabilité d'oscillation $P(\nu_{\mu}\to \nu_e)$ en fonction de l'énergie du neutrino incident en détectant les réactions des neutrinos d'un faisceau initialement composé uniquement de neutrinos muoniques. Le détecteur doit donc être capable de différencier une réaction de neutrino muonique d'une réaction de neutrino électronique, ainsi que de mesurer précisément l'énergie mise en jeu dans ces réactions. Le faisceau de neutrino, dont le principe est décrit en \autoref{sec::faisceau}, va d'abord traverser un détecteur placer proche de la source, typiquement à quelques centaines de mètres, quand la probabilité d'oscillation est encore nulle. Les caractéristiques importantes du faisceau y sont analysées, comme le flux et le spectre en énergie. La possibilité de comparer les observations avant et après oscillation d'une même source de neutrinos est une plus-value des expériences à longue ligne de base utilisant des faisceaux ou des réacteurs. Une plus-value du faisceau est sont ajustabilité : le spectre en énergie peut être modifié précisément, et il est possible de choisir entre un faisceau de neutrino ou d'antineutrino. Bien que la mesure de l'oscillation de neutrino seule suffise théoriquement à mesurer la phase de violation CP, la comparaison à l'oscillation d'antineutrino permet de vérifier le résultat ainsi obtenu.
            
            Le faisceau de neutrino devra traverser en partie la croûte terrestre avant d'atteindre le détecteur lointain. Il est donc nécessaire de prendre en compte les effets de matière décrit en \autoref{sec::violation} dans le calcul de $P(\nu_{\mu}\to \nu_e)$, qui sont différents entre neutrinos et antineutrinos. Cette effet, comme nous allons le montrer en \autoref{sec::param_lbne}, favorise la distinction entre la hiérarchie de masse normale et inversée, sans empêcher une expérience comme \gls{dune} de mesurer la phase de violation CP.
            
            Qu'est-ce que la longue ligne de base, à quoi elle sert\\
            
        \subsection{Le faisceau de neutrino}\label{sec::faisceau}
            Principe d'un faisceau de neutrino, off axis, peak energy, can switch to anti-nu, can tune energy to reach maximum oscillation probability.
            
        \subsection{Paramètres accessibles par les LBNE accélérateur}\label{sec::param_lbne}
                
             Matter effect will artificially change the nu-antinu behavior, but not identically depending on MH->can measure MH. Effet CP est du même ordre de grandeur que effet de matière (il faut considérer les deux pour mesurer CP).
             
        \subsection{Expériences passées, actuelles et futures}
            Décrire un peu les différentes technologies utilisées, présenter les résultats actuels avec un peu plus de détails que dans le chapitre précédent.\\
            +résultats innexplicables par PMNS (Diwan2016)
            T2K, nova, minos, minerva
            \cite{Diwan2016}\\
            DUNE, essneusb, hyperK
            
        
    \section{L'expérience \texorpdfstring{DU$\nu$E}{DUNE}}
    
        \subsection{Introduction}
            \cite{Acciarri2016}\\
            où, par qui, pourquoi, comment

        \subsection{Objectifs scientifiques}
            Donner des chiffres! L/E, flux attendu, graphs...
        
            \cite{Collaboration2015}\\
            - Quelle est l'origine de l'asymétrie matière-antimatière dans l'univers?\\
            - Quelles sont les symétries fondamentales sous-jacentes de l'univers?\\
            - Y a-t-il une grand théorie unifiée de l'univers?\\
            - Comment explosent les supernovae et quelle nouvelle physique pouvons nous apprendre des explosions de neutrinos\footnote{\textit{neutrino burst} en anglais}?\\
            
            Programme scientifique principal :\\ mesures de précision des paramètres gouvernant les oscillations (anti)e vers (anti)mu : phase CP, ordre de masse, theta23\\
            désintégration de proton\\
            ne de supernovae\\
            
            Programme scientifique secondaire:\\
            BSM : interactions non standards, neutrinos stériles, apparition de neutrinos tau\\
            Oscillations via neutrinos atmosphériques\\
            interactions neutrino : section efficace, effets nucléaire, structure des nucléons, thetaW\\
            Matière noire
            
        
        \subsection{Description de l'expérience}
            Tous les détails techniques : faisceau, détecteurs proches et lointains
            -> Redondance avec section suivante?
    
    \section{Le projet \texorpdfstring{protoDU$\nu$E}{protoDUNE}}
    
        \subsection{Introduction}
            Faire le lien avec la section précédente, objectifs, localité du projet, labo impliqués
    
        \subsection{Les Chambres à Projection Temporelle}
            commencer par un petit récap des technologie précédente (chambre à bulle, brouillad...) et lister les avantages de la TPC. Exemples d'expérience utilisant des TPC à gaz (ALICE), finir sur les TPC à liquides.
        
        \subsection{La technologie à double phase d'argon}
            \subsubsection{Avalanche de Townsend}
                chambre à file, toussa toussa.
            \subsubsection{Avantages et inconvénients par rapport à la technologie simple phase}\label{sec::townsend_avalanche}
            \subsubsection{État de l'art}
            \subsubsection{Les points peu étudiés}
                UV réémission
        
        \subsection{Les TPC de \texorpdfstring{protoDU$\nu$E}{protoDUNE}}
            \subsection{le 311}
                pas plus de quelques paragraphe, il sera décrit plus en détail au chapitre 4
            \subsection{le 666}
                un peu plus que sur le 311