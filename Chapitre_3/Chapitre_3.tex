
\chapter{La technologie de Chambre a projection temporelle à argon liquide}

\chapterprecishere{
``Potentielle citation sans aucun rapport avec le sujet"\par\raggedleft--- \textup{Personne inconnue}, contexte à déterminer
}
%    \subsection{Introduction}
%      Faire le lien avec la section précédente, objectifs, localité du projet, labo impliqués
    
  \section{Introduction}
    Le principe de la \acrfull{tpc} a été proposé pour la première fois en 1974 par D.R.~Nygren\cite{Nygren1974} et est illustré en \autoref{fig::tpc}. Il s'agit d'un volume cylindrique rempli de gaz à travers lequel est appliqué un champ électrique. Une particule chargée traversant le milieu gazeux ionise le gaz, laissant des paires électron-ion dans son sillage. Le champ électrique sépare les ions, qui vont dériver vers la cathode, des électrons, qui vont dériver vers l'anode. Les électrons peuvent être amplifier dans le gaz grâce à un fort champ électrique, pouvant augmenter le signal jusqu'à un facteur 1000. La lecture des électrons peut se faire en utilisant plusieurs plans de détections. Avec au moins deux plans perpendiculaire, l'information $x$ et $y$ de la trace peut être reconstruite. L'information $z$ peut être obtenu grâce au temps de dérive et à la vitesse de dérive des électrons dans le gaz. L'énergie déposée par la particule est proportionnelle à la charge collectée, et l'information de la perte d'énergie par unité de longueur permet d'identifier les particules grâce à la formule de Bethe-Bloch (voir \autoref{sec::MPV}). Il est également possible de magnétiser la \gls{tpc} afin d'identifier le signe de la charge des particules.

    La \gls{tpc} a plusieurs avantages:
    \begin{itemize}
      \item[$\bullet$] Toutes les interactions ont lieu dans un volume homogène et continue, sans zones mortes.
      \item[$\bullet$] L'information 3D des traces est reconstruite sans ambiguïté.
      \item[$\bullet$] Il est possible de séparer plusieurs traces même quand la densité de trace est importante, permettant l'étude d'interactions complexes.
      \item[$\bullet$] Il est possible de reconstruire la charge déposée dans le milieu et d'utiliser le dépôt d'énergie par unité de longueur pour identifier les particules.
    \end{itemize}
    
    La première \gls{tpc} a été utilisée dans le complexe de détecteur PEP-4 pour étudier des collisions électrons-positron. La forme cylindrique est la plus adaptée pour les expériences de collisionneur, et la plus grosse \gls{tpc} à gaz actuellement en service est celle de l'expérience ALICE sur le LHC. Pour une expérience de physique des neutrinos, la forme cylindrique n'apporte aucun avantage et le gaz n'est pas un milieu suffisamment dense pour avoir une nombre d'interactions conséquent. En 1977, C.~Rubbia propose donc la \acrfull{lartpc} \cite{Rubbia1977} spécifiquement pour l'étude des neutrinos.
    
  \section{L'argon liquide}\label{sec::lartpc}
      Dense, peu cher, \\
      proto single phase = qq resultats
    
  \section{La technologie à double phase d'argon}
    \subsection{Principe}
    \subsection{Avalanche de Townsend}
      chambre à fils, toussa toussa.
    \subsection{Avantages et inconvénients}
    \subsection{État de l'art}
    
  \section{\texorpdfstring{protoDU$\nu$E}{protoDUNE}}
    \subsection{\texorpdfstring{protoDU$\nu$E}{protoDUNE}-Single Phase}
    \subsection{WA105/\texorpdfstring{protoDU$\nu$E}{protoDUNE}-Double Phase}
      \subsubsection{le 311}
        pas plus de quelques paragraphe, il sera décrit plus en détail au chapitre 4
      \subsubsection{le 666}
        un peu plus que sur le 311