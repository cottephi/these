\def\ds{\displaystyle} 
\def\sty{\scriptstyle} 
\def\Vec{\overrightarrow} 
\def\vep{\varepsilon}
\def\nabla{\bigtriangledown}

\def\E{    {\mathcal{E}}} 
\def\F{    {\mathcal{F}}} 
\def\H{\hat{\mathcal{H}}} 
\def\M{\hat{\mathcal{M}}} 
\def\N{\hat{\mathcal{N}}} 
\def\P{\hat{\mathcal{P}}}
\def\R{    {\mathcal{R}}}
\def\L{    {\mathcal{L}}}
\def\T{\hat{\mathcal{T}}}
\def\K{\hat{\mathcal{K}}}

\def\U{\hat{U}}

\def\ap{{\hat{a}}^{\,+}} 
\def\am{{\hat{a}}^{\, }} 

\def\etap{{\hat{\eta}}^{\,+}} 
\def\etam{{\hat{\eta}}^{\, }} 

\def\be{\boldmath\begin{equation}}
\def\ee{\end{equation}\unboldmath}

\def\beq{\boldmath\begin{eqnarray}}
\def\eeq{\end{eqnarray}\unboldmath}
\def\bs{ \boldmath$}
\def\es{$ \unboldmath}

\def\TOO{$3\times 1\times \SI{1}{\meter\cubed}$}
\def\SSS{$6\times 6\times \SI{6}{\meter\cubed}$}

\DeclareSIUnit\year{yr}
\renewcommand\appendixpagename{Annexes}
\renewcommand\appendixtocname{Annexes}
\newcommand\nuanu{\overset{(-)}{\nu}}
\newcommand\anu{\overline{\nu}}
\newcommand\dune{DU$\nu$E}
\newcommand\protodp{protoDU$\nu$E--DP}
\newcommand\protosp{protoDU$\nu$E--SP} 
\newcommand\threeL{\SI{3}{\liter}}
\newcommand\driftfield{\SI{0.5}{\kilo\volt\per\centi\meter}}

\newtcolorbox[auto counter,number within=chapter]{activitybox}[2][]{%
	fonttitle=\scshape,
	title={Encadré \thetcbcounter -- #2},
	#1
}

\newcommand*\parttitle{}
\let\origpart\part
\renewcommand*{\part}[2][]{%
   \ifx\\#1\\% optional argument not present?
      \origpart{#2}%
      \renewcommand*\parttitle{#2}%
   \else
      \origpart[#1]{#2}%
      \renewcommand*\parttitle{#1}%
   \fi
}

\makeatletter
\newcommand{\extraPartText}[1]{\def\@extraPartText{#1}}
\pretocmd{\@endpart}{\vspace{8ex}\begingroup\flushleft{\@extraPartText}\par\endgroup\let\@extraPartText\relax}{}{}
\makeatother


\newcommand{\PhDTitle}{Le projet WA105 : un prototype de Chambre à Projection Temporelle à Argon Liquide Diphasique utilisant des détecteurs LEMs} 	%% Titre de la thèse / Thesis title
\newcommand{\PhDTitleEN}{The WA105 project : a prototype of Double Phase Liquid Argon Time Projection Chamber using LEMs detectors.}													%% Titre de la thèse en anglais / Thesis title in english
\newcommand{\PhDname}{Philippe Cotte} 															%% Civilité, nom et prénom /  Civility, first name and name 
\newcommand{\NNT}{NNT Number} 															%% Numéro National de Thèse (donnée par la bibliothèque à la suite du 1er dépôt)/ National Thesis Number (given by the Library after the first deposit)

\newcommand{\ecodoctitle}{Particules Hadrons Énergie et Noyau : Instrumentation, Image, Cosmos et Simulation} 													%% Nom de l'ED. Voir site de l'Université Paris-Saclay / Full name of Doctoral School. See Université Paris-Saclay website
\newcommand{\ecodocacro}{PHENIICS}																%% Sigle de l'ED. Voir site de l'Université Paris-Saclay / Acronym of the Doctoral School. See Université Paris-Saclay website
\newcommand{\ecodocnum}{576} 																%% Numéro de l'école doctorale / Doctoral School number
\newcommand{\PhDspeciality}{Physique des particules} 										%% Spécialité de doctorat / Speciality 
\newcommand{\PhDworkingplace}{l'Université Paris-Sud\\
	au Comissariat à l'Énergie Atomique et aux Énergies Alternatives (CEA), \\
	au sein du Département de Physique des Particules (DPhP)\\
	de l'Institut de Recherche sur les lois Fondamentales de l'Univers (IRFU)} 										%% Établissement de préparation / PhD working place : l'Université Paris-Sud, l'Université de Versailles-Saint-Quentin-en-Yvelines, l'Université d'Evry-Val-d'Essonne, l'Institut des sciences et industries du vivant et de l'environnement (AgroParisTech), CentraleSupélec,l'Ecole normale supérieure de Cachan, l'Ecole Polytechnique, l'Ecole nationale supérieure de techniques avancées, l'Ecole nationale de la statistique et de l’administration économique, HEC Paris, l'Institut d'optique théorique et appliquée, Télécom ParisTech, Télécom SudParis   
\newcommand{\defenseplace}{Gif-sur-Yvette} 											%% Ville de soutenance / Place of defense
\newcommand{\defensedate}{17 Septembre 2019} 															%% Date de soutenance / Date of defense

%%% Établissement / Institution
%%% Si la thèse a été produite dans le cadre d'une co-tutelle, commenter la partie "Pas de co-tutelle" et décommenter la partie "Co-tutelle" / If the thesis has been prepared in guardianship, comment the part "Pas de co-tutelle" and uncomment the part "Co-tutelle"

%%%%%%%%%%%%%%%%%%%%%%%%%
%%% Pas de co-tutelle %%%
%%%%%%%%%%%%%%%%%%%%%%%%%

\newcommand{\logoEtt}{blank}																%% NE PAS MODIFIER / DO NOT MODIFY
\newcommand{\vpostt}{0.1} 																	%% NE PAS MODIFIER / DO NOT MODIFY
\newcommand{\hpostt}{6}																		%% NE PAS MODIFIER / DO NOT MODIFY
\newcommand{\logoEt}{UPSUD} 																	%% Logo de l'établissement de soutenance. Indiquer le sigle / Institution logo. Indicate the acronym : AGRO, CENTSUP, ENS, ENSAE, ENSTA, HEC, IOGS, TPT, TSP, UEVE, UPSUD, UVSQ, X 
\newcommand{\vpos}{0.1}																		%% À modifier au besoin pour aligner le logo verticalement / If needed, modify to align logo vertilcally
\newcommand{\hpos}{13}																		%% À modifier au besoin pour aligner le logo horizontalement / If needed, modify to align logo horizontaly

%%%%%%%%%%%%%%%%%%
%%% Co-tutelle %%%
%%%%%%%%%%%%%%%%%%

%\newcommand{\logoEt}{etab} 																%% Logo de l'université partenaire. Placer le fichier .png dans le répertoire '/media/etab' et indiquer le nom du fichier sans l'extension / Logo of partner university. Place the .png file in the directory '/media/etab' and point the file name without the extension
%\newcommand{\vpos}{0.1}																	%% À modifier au besoin pour aligner les logos verticalement / If needed, modify to align logos vertilcally
%\newcommand{\hpos}{11}																		%% À modifier au besoin pour aligner les logos horizontalement / If needed, modify to align logos horizontaly
%\newcommand{\logoEtt}{etab2}  																%% Logo de l'établissement de soutenance. Le nom du fichier correspond au sigle de l'établissement /  Institution logo. Filename correspond to institution acronym : AGRO, CENTSUP, ENS, ENSAE, ENSTA, HEC, IOGS, TPT, TSP, UEVE, UPSUD, UVSQ, X 
%\newcommand{\vpostt}{0.1} 																	%% À modifier au besoin pour aligner les logos verticalement / If needed, modify to align logos vertilcally
%\newcommand{\hpostt}{6}																	%% À modifier au besoin pour aligner les logos horizontalement / If needed, modify to align logos horizontaly


%%% JURY

% Lors du premier dépôt de la thèse le nom du président n’est pas connu, le choix du président se fait par les membres du Jury juste avant la soutenance. La précision est apportée sur la couverture lors du second dépôt / Choice of the jury's president is made during the defense. Thus, it must be specified only for the second file deposition in ADUM.
% Tous les membres du juty listés doivent avoir été présents lors de la soutenance / All the jury members listed here must have been present during the defense.

%%% Membre n°1 (Président) / Member n°1 (President)
\newcommand{\jurynameA}{Alessandra Tonazzo}
\newcommand{\juryadressA}{Professeure, Université Paris-Diderot, Laboratoire
APC}
\newcommand{\juryroleA}{Présidente}

%%% Membre n°2 (Rapporteur) / Member n°2 (Rapporteur)
\newcommand{\jurynameB}{Anselmo Meregaglia}
\newcommand{\juryadressB}{Chargé de recherche, Université de Bordeaux, CENBG}
\newcommand{\juryroleB}{Rapporteur}

%%% Membre n°3 (Rapporteur) / Member n°3 (Rapporteur)
\newcommand{\jurynameC}{Inés Gil Botella}
\newcommand{\juryadressC}{Directrice de recherche, CIEMAT, Madrid}
\newcommand{\juryroleC}{Rapporteure}

%%% Membre n°4 (Examinateur) / Member n°4 (Examinateur)
\newcommand{\jurynameD}{Éric Baussan}
\newcommand{\juryadressD}{Maître de conférence, Université de Strasbourg, IPHC}
\newcommand{\juryroleD}{Examinateur}

%%% Membre n°5 (Examinateur) / Member n°5 (Examinateur)
\newcommand{\jurynameE}{Philippe Schune}
\newcommand{\juryadressE}{Cadre scientifique des EPIC, CEA-Paris Saclay}
\newcommand{\juryroleE}{Examinateur}

%%% Membre n°6 (Directeur de thèse) / Member n°6 (Thesis supervisor)
\newcommand{\jurynameF}{Edoardo Mazzucato}
\newcommand{\juryadressF}{Cadre scientifique des EPIC, CEA-Paris Saclay}
\newcommand{\juryroleF}{Directeur de thèse}

\newcommand{\logoEd}{PHENIICS}																		%% Logo de l'école doctorale. Indiquer le sigle / Doctoral school logo. Indicate the acronym : 2MIB; AAIF; ABIES; BIOSIGNE; CBMS; EDMH; EDOM; EDPIF; EDSP; EOBE; INTERFACES; ITFA; PHENIICS; SDSV; SDV; SHS; SMEMAG; SSMMH; STIC
\newcommand{\keywordsFR}{Physique des neutrinos, WA105, DLArTPC, LEM, DU$\nu$E}														%% Mots clés en français, séprarés par des , / Keywords in french, separated by ,
\newcommand{\abstractFR}{Le projet WA105/\protodp{} est une expérience de prototypage qui a pour objectif de tester la technologie de Chambre à Projection Temporelle à Argon Liquide Diphasique (DLArTPC) à grande échelle dans le but de l'utiliser dans la future expérience de physique des neutrinos \dune{}. Prévue fin 2026 aux USA, \dune{} vise à déterminer l'ordre des masses des neutrinos ainsi que la violation de CP dans le secteur leptonique. Le travail de cette thèse s'oriente dans une premier temps autour des tests et simulations effectués sur les éléments de détection et d'amplification des détecteurs de WA105. Dans un second temps, la thèse s'oriente autour de l'analyse des traces de muons cosmiques vues par un premier prototype de \SI{4}{\tonne}, opéré en 2017 au CERN.\\

La technologie DLArTPC est une variante de la technologie LArTPC permettant une amplification des électrons extraits de la phase liquide à la phase gazeuse. Les amplificateurs d'électrons (LEMs) sont des plaques de PCB de $50\times\SI{50}{\centi\meter\squared}$ épais de \SI{1}{\milli\meter}, percés de \numprint{400000} trous de \SI{500}{\micro\meter} de diamètre, recouvertes de chaque côté par une mince couche de cuivre. Une différence de potentiel de l'ordre de \SI{3}{\kilo\volt} permet d'atteindre un gain supérieur à 10. Une partie du travail de cette thèse a consisté à simuler la dérive des électrons à travers ces LEMs afin d'étudier les efficacités de collection de charge. Une autre partie de cette thèse a consisté à mesurer les caractéristiques importantes (épaisseur, tenue en tension) des amplificateurs destinés au démonstrateur de \SI{300}{\tonne} de WA105, dont la mise en route a été effectuée fin août 2019 au CERN.\\

Le gain est une des caractéristiques principales d'une DLArTPC, et il a été étudié dans le prototype de \SI{4}{\tonne} grâce à la détection de muons cosmiques. Des comparaisons sont effectuées avec les résultats d'un prototype de \SI{3}{\liter} datant de 2014, et un programme de reconstruction de trace dédié a été développé pour traiter certains événements bruités.\\

Le travail effectué dans cette thèse a permis de mieux comprendre le fonctionnement des DLArTPCs, nottament en ce qui concerne l'aspect multiplication et dérive des électrons. Ces connaissances seront importantes lors de l'opération du démonstrateur de \SI{300}{\tonne} au CERN, ainsi que lors de l'exploitation du module DLArTPC de \dune{}.}															%% Résumé en français / abstract in french

\newcommand{\keywordsEN}{Neutrino physics, WA105, DLArTPC, LEM, DU$\nu$E}														%% Mots clés en anglais, séprarés par des , / Keywords in english, separated by ,
\newcommand{\abstractEN}{The WA105/\protodp{} project is a prototyping experiment which goal is to test the Double Phase Liquid Argon Time Projection Chamber (DLArTPC) technology at large scale, to use it in the future neutrinos physics experiment \dune{}. Scheduled for the end of 2026 in the USA, \dune{} aims at measuring the neutrinos mass ordering and the leptonic CP symetry violation.  The first part of this thesis is dedicated to tests and simulations of the detection and amplification elements of the WA105 detectors. The second part is focused on the analysis of cosmic muon tracks seen by a first prototype of \SI{4}{\tonne}, operated at CERN in 2017.\\

The DLArTPC technology is a variation of the LArTPC technology allowing for the amplification of the electrons extracted from the liquid phase to the gas phase. The Large Electron Amplifiers (LEMs) are $50\times\SI{50}{\centi\meter\squared}$ PCB plates with a thickness of \SI{1}{\milli\meter}, pierced by  \numprint{400000} holes of \SI{500}{\micro\meter} diameter, covered on each side by a thin layer of copper giving a gain superior to 10. Part of this thesis work is about the simulation of electrons drifting through those LEMs to study the charge collection efficiencies. Another part of this thesis is about the measurement of important caracteristics (thickness, voltage stability) of the LEMs that are used in the \SI{300}{\tonne} demonstrator of WA105, which commissionning was done in the end of August 2019.\\

The gain is one of the main caracteristics of a DLArTPC, and it has been studied in the \SI{4}{tonne} prototype by detecting cosmic muons. Comparisons are done with previous results from 2014 from a smaller prototype of \SI{3}{\liter}, and a dedicated reconstruction program was created to analyse noisy events.\\

The work done in the thesis allowed for a better understanding of DALrTPCs, mainly on the multiplication and drift of electrons. This knowledge will be important during the operation of the \SI{300}{\tonne} demonstrator at CERN, and during the operationg of the DLArTPC module of \dune{}.}															%% Résumé en anglais / abstract in english