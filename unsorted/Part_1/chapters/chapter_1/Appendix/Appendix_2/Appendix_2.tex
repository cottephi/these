\section{L'étrange nature de la nature}\label{sec::strange_nature}
    
    Commençons par l'exemple type qui introduit la mécanique quantique : l'électron dans la double fente, illustré à la Figure \ref{Fig::double_slit}. La Figure \ref{Fig::double_slit_a} représente l'expérience des fentes d'Young\cite{young}, qui a mit en évidence le caractère ondulatoire de la lumière. En effet, comme l'illustre la Figure \ref{Fig::double_slit_b}, on peut faire l'analogie entre une onde de lumière et une onde à la surface de l'eau, où les creux et les bosses après les fentes créent une figure d'interférence. Il est important de noter ici que même si l'on envoi une \textit{seule onde} (une seule perturbation de la surface de l'eau), \textit{la figure d'interférence existe}. La lumière présentant les mêmes interférences, on en déduit qu'elle se comporte bien comme une onde.
                    
    Si l'on envoi des balles, i.e des particules classiques, (Figure \ref{Fig::double_slit_c}, on s'attend à avoir une distribution en forme de double fente sur le mur. Ce n'est pas ce que l'on observe quand on envoi des électrons. La Figure \ref{Fig::double_slit_d} montre le même résultats qu'avec des photons\cite{double_slit_electrons}, indiquant que les électrons aussi se comportent comme des ondes. Mais il est possible aussi d'envoyer des électrons un par un. Si l'on fait ceci, et si les électrons \textit{sont} des ondes, une figure d'interférence devrait être vue. Ce n'est pas le cas, comme le montre la Figure \ref{Fig::double_slit_e} : un seul point apparaît. Là où les choses se corsent, c'est si l'on envoi des électrons un par un pendant un moment, et qu'on laisse leurs traces à l'écran. Au bout d'un certain nombre de tir, on observe une figure d'interférence. Ceci n'a strictement aucune analogie classique correcte\footnote{La plus proche est celle de la bille en suspension sur de l'eau vibrante. Voir la vidéo de \textit{Veritasium} à ce sujet\cite{veritasium}.}. Et pour compliquer encore les choses, si l'on a un moyen de détecter par quelle fente passe l'électron, alors aucune interférence n'est observée.
                    
    L'interprétation faite de ce phénomène suit la théorie des probabilités. Il n'est pas possible de la \textit{comprendre}, au sens ou on ne peut la mettre en relation avec rien de notre expérience de notre vie de tous les jours, mais elle est en accord avec les observations à un tel point que plus personne ne doute de sa justesse aujourd'hui. L'idée est d'associer à chaque possibilité qu'a l'électron d'arriver en $x$ une amplitude de probabilité (un nombre complexe). L'électron n'a ici que deux possibilités, passer par la fente 1 ou par la fente 2, on nommes les amplitudes correspondantes $\phi_1(x)$ et $\phi_2(x)$. L'addition de ces amplitudes donne une troisième amplitude $\phi(x)$, dont on défini le carré de la norme que étant la probabilité de trouver l'électron en $x$. Tout ceci semble artificiel