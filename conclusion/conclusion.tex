\chapter{Conclusion}

Le travail de cette thèse s'est répartie entre l'analyse des données du prototype de \TOO{} et les tests et caractérisation des \glspl{lem} et des anodes des \glspl{crp} du démonstrateur de \SSS{} du projet \gls{wa105}.

Les tests hautes tensions des \glspl{lem} réalisées dans l'enceinte haute pression du \gls{cea} Saclay, ainsi que les opérations du prototype de \TOO{}, ont permis de mettre en évidence une limitation du modèle initialement choisi pour les \glspl{lem}. Un nouveau modèle comportant des zones mortes plus grandes a été développé et testé afin de tenir des tensions allant jusqu'à \SI{3.2}{\kilo\volt}, correspondant à un gain effectif attendu de 20. Les tests des \glspl{crp} du \SSS{} réalisés dans une boîte cryogénique au \gls{cern} ont permis de valider ces derniers avant de les placer dans le cryostat. Les tensions ont été maintenues aux valeurs nominales d'opération pendant plusieurs jours, et la planéité des \glspl{crp} a été mesurée inférieur à \SI{1.75}{\milli\meter}, ce qui implique une variation du gain sur la surface de ces \glspl{crp} de moins de 1\,\%. Ces \glspl{crp} ont par la suite été installés dans le cryostat du \SSS{}, qui prendra ses premières données en août 2019.

Les simulations réalisées avec \gls{ansys} et Garfield ont permis de mettre en évidence des pertes d'électrons dues aux zones mortes des \glspl{lem}, et ont quantifié pixel par pixel l'efficacité de transmission de ces derniers, pour le modèle CFR-34. Cette même efficacité de transmission a été quantifié pour les bords du modèle CFR-35. Il a en particulier été vu que certains canaux de lectures au niveau des bords des \glspl{lem} verront une fraction de la charge, ce qui peut être pris en compte dans lors de l'analyse des interactions, tandis que d'autres canaux seront totalement aveugles. Ceci implique une perte d'information, qui pourra impacter la reconstruction de vertex d'interaction si ceci venaient à se trouver au niveau de ces canaux. Une étude plus poussées, avec un algorithme de reconstruction complet comme celui de \gls{larsoft}, permettra d'évaluer ces effets.

Les simulations du transport des électrons à travers le \gls{crp} ont permis de mettre en évidence deux phénomènes pouvant entraîner une perte d'électrons. Ces derniers peuvent être perdus après avoir atteint la phase gazeuse mais avant d'atteindre le \gls{lem}, en terminant leur course sur le cuivre du bas du \gls{lem}, ou se retrouver piégés sur les RIMs du haut du \gls{lem}. Le premier phénomène se traduit, une fois combiné à l'efficacité d'extraction du liquide vers le gaz, par un plateau d'efficacité. Si le niveau d'argon est situé au milieu entre la grille d'extraction et le \gls{lem} ces deux étant séparer de \SI{1}{\centi\meter}, ce plateau correspond à des tensions comprises entre \SI{2}{\kilo\volt} et \SI{3}{\kilo\volt}. Ceci est en adéquation avec les mesures faites par le prototype de \threeL{} et se retrouve également avec les données du \TOO{}. Se placer à ces tensions permet de minimiser l'impact sur le gain effectif des potentiels déformations du \gls{crp}, qui vont modifier les champs électriques entre la grille et les \glspl{lem}, réduisant l'impact de ces déformations à celles induite par la variation du température due au gradient présent dans la phase gazeuse. Le second phénomène, la perte d'électrons sur les RIMs, a été simulé pour un \gls{lem} avant charging up. La tendance observée est que l'efficacité de collection augmente avec la tension appliquée entre le haut du \gls{lem} et l'anode, ce qui a été vérifié par des mesures faites dans l'enceinte haute pression, bien que ces mesures aient été faites après charging up. Des différences notables sont toutefois présentes entre la mesure après charging up et la simulation avant charging up, de l'ordre de 15\,\%. La valeur absolue de l'efficacité n'étant pas mesurable dans l'enceinte haute pression, ce sont les valeurs de la simulations qui ont été utilisées par la suite.

Ces simulations ont été utilisées dans l'analyse des données du prototype de \TOO{}, qui n'a pas pu opérer à des tensions fixées à cause d'une limitation de la tension maximale applicable à la grille d'extraction. Les valeurs mesurées du gain effectif ont été ramenées à celles mesurées dans le prototype de \threeL{}, et une différence significative a été observée. Les gains mesurés par le \TOO{} sont inférieurs à ceux mesurés dans le \threeL{}, avant comme après charging up. Ces différences, comprises entre 8\,\% et 35\,\%, peuvent être dues à une mauvaise estimation du comportement et de la valeur de l'efficacité de collection de l'anode en fonction du champ d'induction, comme mentionné dans le paragraphe précédent. De plus, de nombreuses corrections sont appliquées aux valeurs du \TOO{} afin de les comparer aux valeurs du \threeL{}. Des systématiques supplémentaires peuvent jouer un rôle dans mes différences observées. 

Les résultats du \TOO{} ont permis de conclure que la technologie \gls{dlartpc} fonctionne avec une dérive de une mètre et un \gls{crp} de trois mètres carrés. De plus, il a mis en évidence la nécessité de systématiquement tester les \glspl{crp} avant des les introduire dans le cryostat, ce qui a motivé l'étude en boîte cryogénique. Il a également permis de mettre en évidence une limitation de la tenue en tension du modèle de \gls{lem} CFR-34, ce qui a amené au développement du modèle CFR-35.

La mise en route du prototype de \SSS{} est la prochaine étape que le projet \gls{wa105} va franchir. Les performances de ce dernier vont permettre de s'assurer de la capacité de la technologies \gls{dlartpc} à répondre aux besoins de \gls{dune}; sa construction ayant dors et déjà permis d'identifier les challenges inhérents à la construction d'une \gls{dlartpc} à l'échelle de \gls{dune} et de s'y préparer.