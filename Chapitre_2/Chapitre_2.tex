\chapter{L'expérience \texorpdfstring{DU$\nu$E}{DUNE} et son projet de R\&D~: \texorpdfstring{protoDU$\nu$E}{protoDUNE}}
    \chapterprecishere{
        ``Potentielle citation sans aucun rapport avec le sujet"\par\raggedleft--- \textup{Personne inconnue}, contexte à déterminer
    }
    
    Ce chapitre commence par introduire les principales expériences d'oscillation de neutrinos d'accélérateurs à long ligne de base, dont \gls{dune} fait parti, en expliquant leurs points forts et leurs points faibles. Ensuite, une section est dédiée à l'expérience \gls{dune} elle même, puis une autre section est dédiée au projet \gls{wa105} faisant parti des expériences de prototypage de \gls{dune}.
        
    \section{DU$\nu$E, une expériences d'oscillation de neutrinos d'accélérateurs à long ligne de base}
    
        \gls{dune} est une expérience d'oscillation des neutrinos à longue ligne de base, qui enverra un faisceau de neutrino depuis le Fermilab (Illinois) vers l'installation de recherche souterraine de Sanford (Dakota du sud)(voir \autoref{fig::dune}). Les neutrinos, créé par un accélérateur de proton, parcourront $\SI{1300}{\kilo\meter}$ à travers la croûte terrestre avant d'être détectés par quatre \glspl{tpc} de $\SI{10}{\kilo\tonne}$ chacune. Ses objectifs sont multiples, mais les principaux sont la mesure de la phase de violation CP et la détermination de la hiérarchie de masse, en mesurant la probabilité d'apparition $P(\nu_{\mu}\to\nu_e)$ de neutrinos électroniques dans un faisceau initialement constitué de neutrinos muoniques.
        
        Les expériences à longue ligne de base sont caractérisées par le fait de chercher à atteindre un maximum de probabilité d'oscillation (voir \autoref{sec::oscillations}) en plaçant un détecteur loin de la source et en cherchant à détecter des neutrinos d'énergies correspondantes à ce maximum. Dans ces expériences, les neutrinos peuvent être issus de réacteurs nucléaire (comme dans l'expérience JUNO\cite{juno}), du soleil (expérience SNO\cite{sno}), de réactions de rayons cosmiques dans l'atmosphère (expérience Super Kamiokande\cite{skk_atm}), de supernovae (expérience Super Kamiokande\cite{skk_atm}) ou de faisceaux créés par l'homme. Dans cette section, nous nous intéressons au dernier cas, qui est celui de \gls{dune}.
    
        \subsection{Qu'est ce qu'une expérience d'oscillation de neutrinos d'accélérateurs à longue ligne de base?}
            
            Comme il a été discuté en \autoref{sec::oscillations}, la probabilité qu'a un neutrino de changer de saveur oscille avec le ratio $L/E$, où $L$ est la distance entre le point d'origine du neutrino et l'endroit où il est détecté (ligne de base), et $E$ est l'énergie de ce neutrino. Ce phénomène étant oscillant, il existe des ratios $L/E$ privilégiés pour la détection des changements de saveur, où la probabilité de changement est localement maximale. Les expériences cherchant à détecter les oscillations des neutrinos cherchent généralement à se placer à une ligne de base tel que $L/E$ Corresponde à un de ces maxima. L'expérience \gls{dune} sera capable de couvrir deux de ces maxima ce qui permettra à la fois de déterminer la hiérarchie de masse et de mesurer la phase de violation CP (voir \autoref{sec::dune}). 
            
            Les expériences d'oscillation de neutrinos d'accélérateurs à longues lignes de base cherchent à mesurer la probabilité d'oscillation $P(\nu_{\mu}\to \nu_e)$ en fonction de l'énergie du neutrino incident en détectant les réactions des neutrinos d'un faisceau initialement composé uniquement de neutrinos muoniques. Le détecteur doit donc être capable de différencier une réaction de neutrino muonique d'une réaction de neutrino électronique, ainsi que de mesurer précisément l'énergie mise en jeu dans ces réactions. Le faisceau de neutrino, dont le principe est décrit en \autoref{sec::faisceau}, va d'abord traverser un détecteur placer proche de la source, typiquement à quelques centaines de mètres, quand la probabilité d'oscillation est encore nulle. Les caractéristiques importantes du faisceau y sont analysées, comme le flux et le spectre en énergie. La possibilité de comparer les observations avant et après oscillation d'une même source de neutrinos est une plus-value des expériences à longue ligne de base utilisant des faisceaux ou des réacteurs. Une plus-value du faisceau est sont ajustabilité en flux et en énergie. Le spectre en énergie peut être choisie précisément, dans une gamme d'énergie allant de \SI{0.5}{\giga\electronvolt} à \SI{100}{\giga\electronvolt}, et il est possible de choisir entre un faisceau de neutrino ou d'anti-neutrino. Bien que la mesure de l'oscillation de neutrino seule suffise théoriquement à mesurer la phase de violation CP, la comparaison à l'oscillation d'anti-neutrino permet de vérifier le résultat ainsi obtenu.
            
            Le faisceau de neutrino devra traverser en partie la croûte terrestre avant d'atteindre le détecteur lointain. Il est donc nécessaire de prendre en compte les effets de matière décrit en \autoref{sec::violation} dans le calcul de $P(\nu_{\mu}\to \nu_e)$, qui sont différents entre neutrinos et anti-neutrinos. Cette effet, comme nous allons le montrer en \autoref{sec::param_lbne}, favorise la distinction entre la hiérarchie de masse normale et inversée, sans empêcher une expérience comme \gls{dune} de mesurer la phase de violation CP.
            
        \subsection{Le faisceau de neutrino}\label{sec::faisceau}
        
            Il existe trois principaux types de faisceau de neutrino :
            \begin{itemize}
                \item les faisceaux $\beta$, qui récupèrent les neutrinos issues de désintégrations $\beta$,
                \item les usines à neutrinos, qui récupèrent les neutrinos issues de la désintégrations de muons
                \item les faisceaux mésoniques, qui récupèrent les neutrinos issues de la désintégrations de méson.
            \end{itemize}
            Seul la dernière de ces trois technologies a déjà été utilisé pour des expériences, les deux premières étant encore en phase de prototypage. L'expérience \gls{dune} utilisera un faisceau mésonique, aussi nous nous concentrerons sur cette technologie dans la suite du texte. Pour plus d'information sur les faisceaux $\beta$, le lecteur peut consulter l'article~\cite{Wildner2012}, et l'article~\cite{Bogomilov2014} pour plus d'information sur les usines à neutrinos.
            
            %Le faisceau $\beta$, le plus récent des trois, est encore en phase de prototypage. Nous n'en parlons que rapidement, par soucis d'exhaustivité, car ce n'est pas la technologie employée dans \gls{dune}. Plus de détails sont consultables dans cet article~\cite{Wildner2012}. Il s'agit de produire et stocker des noyaux radioactifs se désintégrant via la désintégration $\beta$ puis de les collimater pour de les laisser se désintégrer et ainsi produire un faisceau de (anti)neutrino électronique. L'avantage de cette technologie est d'avoir un spectre en énergie très restreint et ajustable, possiblement de basse énergie permettant de placer un détecteur relativement proche de la source pour correspondre à un maximum de probabilité d'oscillation, limitant ainsi les effets de matière sur les oscillations. Ceci permet d'être plus sensible à la violation CP qu'avec une expérience à longue ligne de base.
            
            %La seconde technologie de faisceau de neutrino est l'usine à neutrino. Comme pour le faisceau $\beta$, elle est en cours de développement et ne sera pas utilisé dans \gls{dune}, nous n'en parlons donc que rapidement, plus de détails se trouvant ici~\cite{Bogomilov2014}. Il s'agit de produire des mésons en projetant des protons sur une cible, de laisser ces mésons se désintégrer en muons et anti-muons qui sont stockés dans un anneau afin de les laisser se désintégrer à leur tour en neutrinos et anti-neutrinos électroniques et muoniques. L'avantage de cette technologie est de produire des faisceaux composés précisément de 50\% de neutrinos muonique et de 50\% de neutrinos électronique, contrairement à la technologie actuelle qui, comme décrit plus bas, produit des faisceaux avec environ 99\% de neutrinos muoniques et environ 1\% de contamination de neutrinos électroniques. En effet la connaissance précise de la composition en saveurs du flux de neutrino est primordiale pour effectuer des mesures d'oscillation de saveurs.
            
            %La troisième technologie, schématisée en \autoref{fig::faisceau}, est celle qui est utilisée actuellement dans les expériences d'oscillation de neutrinos d'accélérateurs à longue ligne de base, et qui est prévue pour \gls{dune}. 
            La \autoref{fig::faisceau} est un schéma de la création d'un faisceau de neutrinos par désintégration de mésons. Il s'agit de produire des mésons (essentiellement des pions et des kaons) en envoyant des protons sur une cible fixe, généralement du graphite. Ces mésons sont ensuite focalisés par une ou plusieurs cornes magnétiques, qui permettent également d'enlever les mésons chargés positivement ou négativement suivant que l'on cherche à produire un faisceau de neutrinos ou d'anti-neutrinos. Les mésons restant sont alors envoyés dans un long tuyau (pouvant être de l'ordre du kilomètre) pour s'y désintégrer en muons et neutrinos muoniques, avec une faible contaminations de neutrinos électronique due au canal de désintégration rare du kaon $K$ $K^+ \to e^+ \pi^0 \nu_e$. Si l'énergie est suffisante, il peut également y avoir une faible contamination en neutrinos tauiques due à la désintégrations de mésons $D_s$. Les mésons restant ainsi que les produits de désintégrations autres que les neutrinos sont arrêtés par une couche épaisse, en acier dans l'expérience \gls{dune}. Il est également possible de se passer de tuyau de désintégration et d'envoyer directement les mésons dans la roche ou le béton, afin qu'ils interagissent avant de se désintégrer. Les neutrinos viendront alors des produits de ces interactions à court temps de vie. Le faisceau résultant sera bien moins intense, mais sera composé d'autant de neutrinos que d'anti-neutrinos, électroniques et muoniques, ainsi que d'une fraction non négligeable de neutrinos et d'anti-neutrinos tauiques. Cette technique, appelée \textit{beam dump}, n'est pas utilisée dans \gls{dune}, qui cherche à avoir des neutrinos d'une seule saveur dans son faisceau.
            
            Les articles \cite{Levy2010,McDonald2001,Itow2001} montrent que le spectre en énergie attendue des neutrinos dépend de l'angle entre la direction de propagation du neutrino et l'angle principal du faisceau. Positionner un détecteur de manière légèrement désaxé (\SI{2.5}{\degree} pour l'expérience T2K par exemple) permet non seulement de choisir le maximum du spectre à une énergie correspondant à un maximum de probabilité d'oscillation, mais aussi d'avoir un spectre plus étroit. Dans \gls{dune}, l'idée est d'avoir un spectre assez large pour couvrir deux maxima de probabilité, aussi les détecteurs proche et lointain seront situés sur l'axe du faisceau.
            
            Le faisceau de proton servant à produire les neutrinos pour \gls{dune} sera créé par le complexe d'accélérateur du Fermilab, qui aura été mis à niveau pour fournir une puissance de $\SI{1.2}{\mega\watt}$ pendant les cinq premières année. Une mise à niveau pourra monter cette puissance à $\SI{2.4}{\mega\watt}$. Il fonctionnera par cycle produisant chacun \numprint{1.0e-5} protons de $\SI{60}{\giga\electronvolt}$ (cycle de \SI{0.7}{\second})  à $\SI{120}{\giga\electronvolt}$ (cycle de \SI{1.2}{\second}). Les protons iront frapper une cible longue de $\SI{95}{\centi\meter}$ faite de 47 segments de graphites, où 85\;\% des protons interagiront pour produire des en majorité des pions et des kaons. Les 50 derniers centimètres de la cible sont dans la première corne magnétique (voir \autoref{fig::horn1}), longue de $\SI{3.36}{\meter}$ et large de $\SI{16.5}{\centi\meter}$. Elle sera parcourue par un courant de $\SI{230}{\kilo\ampere}$. L'entrée de la seconde corne est située à $\SI{3.24}{\meter}$ de la sortie de la première corne. Cette seconde corne, parcourue par le même courant, est longue de  $\SI{3.63}{\meter}$ et large de $\SI{39.5}{\centi\meter}$. Chacune des deux cornes est constituée de deux conducteurs, intérieurs et extérieur, faits en Aluminium Al~6061-T6. Les conducteurs extérieurs sont cylindriques, tandis que les conducteurs intérieurs sont fait de deux paraboles communiquant par une petite ouverture de $\SI{9}{\milli\meter}$ dans la première corne et de $\SI{39}{\milli\meter}$ dans la deuxième. Les cornes magnétiques ont été imaginées par S.Van Der Meer en 1961\cite{VanDerMeer1961}.
            
            Dans une corne, le courant entre par le conducteur intérieur et ressort par le conducteur extérieur, créant un champ magnétique entre les deux conducteurs, mais pas à l'intérieur même du volume délimité par le conducteur intérieur. Les particules d'un signe de charge électrique donné produites dans la cible sont libres de traverser le conducteur intérieur -- qui doit être assez fin pour ne pas en absorber trop et assez épais pour supporter le champ magnétique -- mais seront ensuite réfléchis vers l'axe de la corne par le champ magnétique. Les particules de signe opposé seront déviées vers l'extérieur de la corne et s'échapperont de la corne. S.Van Der Meer a montré que les particules feront un angle plus petit avec l'axe de la corne après chaque réflexion, concentrant ainsi le faisceau. Une corne plus longue, ou plusieurs cornes à la suite, permettent une meilleure concentration. Un courant plus grand permet de focaliser des particules plus énergétiques. Le sens du courant permet de choisir entre focaliser les particules chargées positivement ou négativement.
            %Détailler windows?
            
            $\SI{17}{\meter}$ après la sortie de la seconde corne se trouve l'entrée du tuyau de désintégration d'un diamètre de $\SI{4}{\meter}$, long de $\SI{194}{\meter}$ et rempli d'hélium, où les mésons pourront se désintégrer et produire les neutrinos. Ce tuyau est pointé vers le bas, avec une pente de 10~\%. La \autoref{fig::decay_pipe} montre une vue en coupe de face de ce tuyau. A la fin de ce tuyau se trouve la cavité contenant l'absorbeur, à $\SI{30}{\meter}$ sous la surface. C'est un assemblage d'aluminium, d'acier et de béton, où les particules chargées restantes seront arrêtées. A cet endroit se trouve aussi une alcôve à muons dont le but est de détecter les muons produits avec les neutrinos lors des désintégrations des mésons. En effet, ces muons nous donnent une information sur la direction, l'intensité, le flux et la largeur du faisceau.
            
            Le faisceau est alors constitué uniquement de neutrinos, et va continuer en ligne droite jusqu'au détecteur proche, situé à $\SI{300}{\meter}$ de l'absorbeur et $\SI{60}{\meter}$ sous la surface.
            
            TODO: détails sur le ND
            
            Les neutrinos issues de ce faisceau auront des énergies comprises entre $\SI{0.5}{\giga\electronvolt}$ et $\SI{5}{\giga\electronvolt}$. Au niveau du détecteur lointain, $\SI{1300}{\kilo\meter}$ plus loin, les pics de probabilité d'oscillation correspondront à des énergies de $\SI{0.8}{\giga\electronvolt}$ et $\SI{2.4}{\giga\electronvolt}$, qui sont donc comprises dans le spectre du faisceau de neutrinos. Les flux attendues avec et sans oscillations au niveau du détecteur lointains sont présenté en \autoref{fig::nu_flux}.
            
            
        \subsection{Paramètres accessibles par les LBNE accélérateur}\label{sec::param_lbne}
                
             Matter effect will artificially change the nu-antinu behavior, but not identically depending on MH->can measure MH. Effet CP est du même ordre de grandeur que effet de matière (il faut considérer les deux pour mesurer CP).
             
        \subsection{Expériences passées, actuelles et futures}
            Décrire un peu les différentes technologies utilisées, présenter les résultats actuels avec un peu plus de détails que dans le chapitre précédent.\\
            +résultats innexplicables par PMNS (Diwan2016)
            T2K, nova, minos, minerva
            \cite{Diwan2016}\\
            DUNE, essneusb, hyperK
            
        
    \section{L'expérience \texorpdfstring{DU$\nu$E}{DUNE}}\label{sec::dune}
    
        \subsection{Introduction}
            \cite{Acciarri2016}\\
            où, par qui, pourquoi, comment

        \subsection{Objectifs scientifiques}
            Donner des chiffres! L/E, flux attendu, graphs...
        
            \cite{Collaboration2015}\\
            - Quelle est l'origine de l'asymétrie matière-antimatière dans l'univers?\\
            - Quelles sont les symétries fondamentales sous-jacentes de l'univers?\\
            - Y a-t-il une grand théorie unifiée de l'univers?\\
            - Comment explosent les supernovae et quelle nouvelle physique pouvons nous apprendre des explosions de neutrinos\footnote{\textit{neutrino burst} en anglais}?\\
            
            Programme scientifique principal :\\ mesures de précision des paramètres gouvernant les oscillations (anti)e vers (anti)mu : phase CP, ordre de masse, theta23\\
            désintégration de proton\\
            ne de supernovae\\
            
            Programme scientifique secondaire:\\
            BSM : interactions non standards, neutrinos stériles, apparition de neutrinos tau\\
            Oscillations via neutrinos atmosphériques\\
            interactions neutrino : section efficace, effets nucléaire, structure des nucléons, thetaW\\
            Matière noire
            
        
        \subsection{Description de l'expérience}
            Tous les détails techniques : faisceau, détecteurs proches et lointains
            -> Redondance avec section suivante?
    
    \section{Le projet \texorpdfstring{protoDU$\nu$E}{protoDUNE}}
    
        \subsection{Introduction}
            Faire le lien avec la section précédente, objectifs, localité du projet, labo impliqués
    
        \subsection{Les Chambres à Projection Temporelle}
            commencer par un petit récap des technologie précédente (chambre à bulle, brouillad...) et lister les avantages de la TPC. Exemples d'expérience utilisant des TPC à gaz (ALICE), finir sur les TPC à liquides.
        
        \subsection{La technologie à double phase d'argon}
            \subsubsection{Avalanche de Townsend}
                chambre à file, toussa toussa.
            \subsubsection{Avantages et inconvénients par rapport à la technologie simple phase}\label{sec::townsend_avalanche}
            \subsubsection{État de l'art}
            \subsubsection{Les points peu étudiés}
                UV réémission
        
        \subsection{Les TPC de \texorpdfstring{protoDU$\nu$E}{protoDUNE}}
            \subsection{le 311}
                pas plus de quelques paragraphe, il sera décrit plus en détail au chapitre 4
            \subsection{le 666}
                un peu plus que sur le 311