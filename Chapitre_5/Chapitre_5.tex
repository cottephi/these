\chapter{Analyse des performances du \texorpdfstring{$3\times 1\times 1$}{3x1x1}}
    \chapterprecishere{
        ``Potentielle citation sans aucun rapport avec le sujet"\par\raggedleft--- \textup{Personne inconnue}, contexte à déterminer
    }
    
    \section{Introduction technique}
        \subsection{Objectifs}
            La nécessité de l'étape 311 et les mesures à effectuer avec (stabilité d'un grand CRP, uniformité du gain, charging up, tenu en tension...)
        \subsection{La technologie cryogénique et le cryostat}
        \subsection{L'argon}
        \subsection{La cage de dérive et la haute tension}
        \subsection{Les modes de déclenchement}
            \subsubsection{Le CRT}
            \subsubsection{Les PMT}
        \subsection{Le CRP (ce qui est spécifique au 311, puisque le CRP en général est décrit plus tôt)}
        \subsection{Les différents capteurs}
            \subsubsection{Pression}
            \subsubsection{Température}
            \subsubsection{Niveau du liquide}
            \subsubsection{Niveau du CRP}
        \subsection{L'électronique de lecture}
        \subsection{La mise en route}
        
    \section{Les données collectées}
        \subsection{Données de calibrations}
            \subsubsection{Données de piédestaux}
            \subsubsection{Données lumière}
            \subsubsection{Pulsing (à traduire)}
        \subsection{Données de performance}
            \subsubsection{Liste des runs et conditions de tension}
            \subsubsection{Pression et température durant la période d'acquisition}
                \paragraph{Pression}
                \paragraph{Température}
                \paragraph{Influence attendue sur le gain}
        
    \section{Les algorithmes de reconstruction et d'analyse}
        \subsection{L'analyse en direct et stockage des données}
        \subsection{L'analyse avec LArSoft}\label{sec::larsoft}
            \subsubsection{Repérage des "hits"}
            \subsubsection{Amas}
            \subsubsection{Traces}
        \subsection{L'analyse avec QScan}\label{sec::qscan}
            \subsubsection{Repérage des "hits"}
            \subsubsection{Amas}
            \subsubsection{Traces}
        \subsection{Un troisième logiciel d'analyse léger en python}\label{sec::rawdatasoft}
            \subsubsection{Repérage des "hits"}
            \subsection{Transformation de hough}
                -> En annexe si j'ai le temps : proposition d'une variante
            \subsubsection{Traces}
        \subsection{Comparaison des trois logiciels}
        
    \section{Performances de 311}
        \subsection{Pureté}
        \subsection{Chargement (charging up, meilleure traduction à trouver?)}
        \subsection{Gain}
            \subsubsection{Gain dans le CRP}
            \subsubsection{Gain dans un LEM}
            \subsubsection{Gain vs fields}
                \paragraph{Ampli}
                \paragraph{Extraction}
                \paragraph{Induction}
                \paragraph{Drift}
    
    \section{Performances attendues du 666}