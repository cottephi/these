\chapter{L'expérience DU$\nu$E et son projet de R\&D~: protoDU$\nu$E}
    \chapterprecishere{
        ``Potentielle citation sans aucun rapport avec le sujet"\par\raggedleft--- \textup{Personne inconnue}, contexte à déterminer
    }
    
    \section{Les expériences d'oscillation de neutrinos d'accélérateurs à long ligne de base, actuelles et futures}
    
        \subsection{Paramètres accessibles par les LBNE accélérateur}
            
            Qu'est-ce que la longue ligne de base, à quoi elle sert\\
            Principe d'un faisceau de neutrino, off axis, peak energy, can switch to anti-nu, can tune energy to reach maximum oscillation probability. Matter effect will artificially change the nu-antinu behavior, but not identically depending on MH->can measure MH. Effet CP est du même ordre de grandeur que effet de matière (il faut considérer les deux pour mesurer CP).
                
        \subsection{Expériences passées, actuelles et futures}
            Décrire un peu les différentes technologies utilisées, présenter les résultats actuels avec un peu plus de détails que dans le chapitre précédent.\\
            +résultats innexplicables par PMNS (Diwan2016)
            T2K, nova, minos, minerva
            \cite{Diwan2016}\\
            DUNE, essneusb, hyperK
            
        
    \section{L'expérience DU$\nu$E}
    
        \subsection{Introduction}
            \cite{lbnf_vol1}\\
            où, par qui, pourquoi, comment

        \subsection{Objectifs scientifiques}
            Donner des chiffres! L/E, flux attendu, graphs...
        
            \cite{lbnf_vol2}\\
            - Quelle est l'origine de l'asymétrie matière-antimatière dans l'univers?\\
            - Quelles sont les symétries fondamentales sous-jacentes de l'univers?\\
            - Y a-t-il une grand théorie unifiée de l'univers?\\
            - Comment explosent les supernovae et quelle nouvelle physique pouvons nous apprendre des explosions de neutrinos\footnote{\textit{neutrino burst} en anglais}?\\
            
            Programme scientifique principal :\\ mesures de précision des paramètres gouvernant les oscillations (anti)e vers (anti)mu : phase CP, ordre de masse, theta23\\
            désintégration de proton\\
            ne de supernovae\\
            
            Programme scientifique secondaire:\\
            BSM : interactions non standards, neutrinos stériles, apparition de neutrinos tau\\
            Oscillations via neutrinos atmosphériques\\
            interactions neutrino : section efficace, effets nucléaire, structure des nucléons, thetaW\\
            Matière noire
            
        
        \subsection{Description de l'expérience}
            Tous les détails techniques : faisceau, détecteurs proches et lointains
            -> Redondance avec section suivante?
    
    \section{Le projet protoDU$\nu$E}
    
        \subsection{Introduction}
            Faire le lien avec la section précédente, objectifs, localité du projet, labo impliqués
    
        \subsection{Les Chambres à Projection Temporelle}
            commencer par un petit récap des technologie précédente (chambre à bulle, brouillad...) et lister les avantages de la TPC. Exemples d'expérience utilisant des TPC à gaz (ALICE), finir sur les TPC à liquides.
        
        \subsection{La technologie à double phase d'argon}
            \subsubsection{Avalanche de Townsend}
                chambre à file, toussa toussa.
            \subsubsection{Avantages et inconvénients par rapport à la technologie simple phase}\label{sec::townsend_avalanche}
            \subsubsection{État de l'art}
            \subsubsection{Les points peu étudiés}
                UV réémission
        
        \subsection{Les TPC de protoDU$\nu$E}
            \subsection{le 311}
                pas plus de quelques paragraphe, il sera décrit plus en détail au chapitre 4
            \subsection{le 666}
                un peu plus que sur le 311