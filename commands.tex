\def\ds{\displaystyle} 
\def\sty{\scriptstyle} 
\def\Vec{\overrightarrow} 
\def\vep{\varepsilon}
\def\nabla{\bigtriangledown}

\def\E{    {\mathcal{E}}} 
\def\F{    {\mathcal{F}}} 
\def\H{\hat{\mathcal{H}}} 
\def\M{\hat{\mathcal{M}}} 
\def\N{\hat{\mathcal{N}}} 
\def\P{\hat{\mathcal{P}}}
\def\R{    {\mathcal{R}}}
\def\L{    {\mathcal{L}}}
\def\T{\hat{\mathcal{T}}}
\def\K{\hat{\mathcal{K}}}

\def\U{\hat{U}}

\def\ap{{\hat{a}}^{\,+}} 
\def\am{{\hat{a}}^{\, }} 

\def\etap{{\hat{\eta}}^{\,+}} 
\def\etam{{\hat{\eta}}^{\, }} 

\def\be{\boldmath\begin{equation}}
\def\ee{\end{equation}\unboldmath}

\def\beq{\boldmath\begin{eqnarray}}
\def\eeq{\end{eqnarray}\unboldmath}
\def\bs{ \boldmath$}
\def\es{$ \unboldmath}

\def\TOO{$3\times 1\times \SI{1}{\meter\cubed}$}
\def\SSS{$6\times 6\times \SI{6}{\meter\cubed}$}

\DeclareSIUnit\year{yr}
\renewcommand\appendixpagename{Annexes}
\renewcommand\appendixtocname{Annexes}
\newcommand\nuanu{\overset{(-)}{\nu}}
\newcommand\anu{\overline{\nu}}
\newcommand\protodp{protoDU$\nu$E--DP}
\newcommand\protosp{protoDU$\nu$E--SP} 
\newcommand\threeL{\SI{3}{\liter}}
\newcommand\driftfield{\SI{0.5}{\kilo\volt\per\centi\meter}}

\newtcolorbox[auto counter,number within=chapter]{activitybox}[2][]{%
	fonttitle=\scshape,
	title={Encadré \thetcbcounter -- #2},
	#1
}

\newcommand*\parttitle{}
\let\origpart\part
\renewcommand*{\part}[2][]{%
   \ifx\\#1\\% optional argument not present?
      \origpart{#2}%
      \renewcommand*\parttitle{#2}%
   \else
      \origpart[#1]{#2}%
      \renewcommand*\parttitle{#1}%
   \fi
}

\makeatletter
\newcommand{\extraPartText}[1]{\def\@extraPartText{#1}}
\pretocmd{\@endpart}{\vspace{8ex}\begingroup\flushleft{\@extraPartText}\par\endgroup\let\@extraPartText\relax}{}{}
\makeatother


\newcommand{\PhDTitle}{Le projet WA105 : un prototype de Chambre à Projection Temporelle à Argon Liquide Diphasique utilisant des détecteurs LEMs} 	%% Titre de la thèse / Thesis title
\newcommand{\PhDTitleEN}{The WA105 project : a prototype of Double Phase Liquid Argon Time Projection Chamber using LEMs detectors.}													%% Titre de la thèse en anglais / Thesis title in english
\newcommand{\PhDname}{Philippe Cotte} 															%% Civilité, nom et prénom /  Civility, first name and name 
\newcommand{\NNT}{NNT Number} 															%% Numéro National de Thèse (donnée par la bibliothèque à la suite du 1er dépôt)/ National Thesis Number (given by the Library after the first deposit)

\newcommand{\ecodoctitle}{Particules Hadrons Énergie et Noyau : Instrumentation, Image, Cosmos et Simulation} 													%% Nom de l'ED. Voir site de l'Université Paris-Saclay / Full name of Doctoral School. See Université Paris-Saclay website
\newcommand{\ecodocacro}{PHENIICS}																%% Sigle de l'ED. Voir site de l'Université Paris-Saclay / Acronym of the Doctoral School. See Université Paris-Saclay website
\newcommand{\ecodocnum}{576} 																%% Numéro de l'école doctorale / Doctoral School number
\newcommand{\PhDspeciality}{Physique des particules} 										%% Spécialité de doctorat / Speciality 
\newcommand{\PhDworkingplace}{l'Université Paris-Sud\\
	au Comissariat à l'Énergie Atomique et aux Énergies Alternatives (CEA), \\
	au sein du Département de Physique des Particules (DPhP)\\
	de l'Institut de Recherche sur les lois Fondamentales de l'Univers (IRFU)} 										%% Établissement de préparation / PhD working place : l'Université Paris-Sud, l'Université de Versailles-Saint-Quentin-en-Yvelines, l'Université d'Evry-Val-d'Essonne, l'Institut des sciences et industries du vivant et de l'environnement (AgroParisTech), CentraleSupélec,l'Ecole normale supérieure de Cachan, l'Ecole Polytechnique, l'Ecole nationale supérieure de techniques avancées, l'Ecole nationale de la statistique et de l’administration économique, HEC Paris, l'Institut d'optique théorique et appliquée, Télécom ParisTech, Télécom SudParis   
\newcommand{\defenseplace}{Gif-sur-Yvette} 											%% Ville de soutenance / Place of defense
\newcommand{\defensedate}{17 Septembre 2019} 															%% Date de soutenance / Date of defense

%%% Établissement / Institution
%%% Si la thèse a été produite dans le cadre d'une co-tutelle, commenter la partie "Pas de co-tutelle" et décommenter la partie "Co-tutelle" / If the thesis has been prepared in guardianship, comment the part "Pas de co-tutelle" and uncomment the part "Co-tutelle"

%%%%%%%%%%%%%%%%%%%%%%%%%
%%% Pas de co-tutelle %%%
%%%%%%%%%%%%%%%%%%%%%%%%%

\newcommand{\logoEtt}{blank}																%% NE PAS MODIFIER / DO NOT MODIFY
\newcommand{\vpostt}{0.1} 																	%% NE PAS MODIFIER / DO NOT MODIFY
\newcommand{\hpostt}{6}																		%% NE PAS MODIFIER / DO NOT MODIFY
\newcommand{\logoEt}{UPSUD} 																	%% Logo de l'établissement de soutenance. Indiquer le sigle / Institution logo. Indicate the acronym : AGRO, CENTSUP, ENS, ENSAE, ENSTA, HEC, IOGS, TPT, TSP, UEVE, UPSUD, UVSQ, X 
\newcommand{\vpos}{0.1}																		%% À modifier au besoin pour aligner le logo verticalement / If needed, modify to align logo vertilcally
\newcommand{\hpos}{13}																		%% À modifier au besoin pour aligner le logo horizontalement / If needed, modify to align logo horizontaly

%%%%%%%%%%%%%%%%%%
%%% Co-tutelle %%%
%%%%%%%%%%%%%%%%%%

%\newcommand{\logoEt}{etab} 																%% Logo de l'université partenaire. Placer le fichier .png dans le répertoire '/media/etab' et indiquer le nom du fichier sans l'extension / Logo of partner university. Place the .png file in the directory '/media/etab' and point the file name without the extension
%\newcommand{\vpos}{0.1}																	%% À modifier au besoin pour aligner les logos verticalement / If needed, modify to align logos vertilcally
%\newcommand{\hpos}{11}																		%% À modifier au besoin pour aligner les logos horizontalement / If needed, modify to align logos horizontaly
%\newcommand{\logoEtt}{etab2}  																%% Logo de l'établissement de soutenance. Le nom du fichier correspond au sigle de l'établissement /  Institution logo. Filename correspond to institution acronym : AGRO, CENTSUP, ENS, ENSAE, ENSTA, HEC, IOGS, TPT, TSP, UEVE, UPSUD, UVSQ, X 
%\newcommand{\vpostt}{0.1} 																	%% À modifier au besoin pour aligner les logos verticalement / If needed, modify to align logos vertilcally
%\newcommand{\hpostt}{6}																	%% À modifier au besoin pour aligner les logos horizontalement / If needed, modify to align logos horizontaly


%%% JURY

% Lors du premier dépôt de la thèse le nom du président n’est pas connu, le choix du président se fait par les membres du Jury juste avant la soutenance. La précision est apportée sur la couverture lors du second dépôt / Choice of the jury's president is made during the defense. Thus, it must be specified only for the second file deposition in ADUM.
% Tous les membres du juty listés doivent avoir été présents lors de la soutenance / All the jury members listed here must have been present during the defense.

%%% Membre n°1 (Président) / Member n°1 (President)
\newcommand{\jurynameA}{Alessandra Tonazzo}
\newcommand{\juryadressA}{Professeure, Université Paris-Diderot, Laboratoire
APC}
\newcommand{\juryroleA}{Présidente}

%%% Membre n°2 (Rapporteur) / Member n°2 (Rapporteur)
\newcommand{\jurynameB}{Anselmo Meregaglia}
\newcommand{\juryadressB}{Chargé de recherche, Université de Bordeaux, CENBG}
\newcommand{\juryroleB}{Rapporteur}

%%% Membre n°3 (Rapporteur) / Member n°3 (Rapporteur)
\newcommand{\jurynameC}{Inés Gil Botella}
\newcommand{\juryadressC}{Directrice de recherche, CIEMAT, Madrid}
\newcommand{\juryroleC}{Rapporteure}

%%% Membre n°4 (Examinateur) / Member n°4 (Examinateur)
\newcommand{\jurynameD}{Éric Baussan}
\newcommand{\juryadressD}{Maître de conférence, Université de Strasbourg, IPHC}
\newcommand{\juryroleD}{Examinateur}

%%% Membre n°5 (Examinateur) / Member n°5 (Examinateur)
\newcommand{\jurynameE}{Philippe Schune}
\newcommand{\juryadressE}{Cadre scientifique des EPIC, CEA-Paris Saclay}
\newcommand{\juryroleE}{Examinateur}

%%% Membre n°6 (Directeur de thèse) / Member n°6 (Thesis supervisor)
\newcommand{\jurynameF}{Edoardo Mazzucato}
\newcommand{\juryadressF}{Cadre scientifique des EPIC, CEA-Paris Saclay}
\newcommand{\juryroleF}{Directeur de thèse}

\newcommand{\logoEd}{PHENIICS}																		%% Logo de l'école doctorale. Indiquer le sigle / Doctoral school logo. Indicate the acronym : 2MIB; AAIF; ABIES; BIOSIGNE; CBMS; EDMH; EDOM; EDPIF; EDSP; EOBE; INTERFACES; ITFA; PHENIICS; SDSV; SDV; SHS; SMEMAG; SSMMH; STIC
\newcommand{\keywordsFR}{Physique des neutrinos, WA105, DLArTPC, LEM, DUNE}														%% Mots clés en français, séprarés par des , / Keywords in french, separated by ,
\newcommand{\abstractFR}{Le projet WA105/ProtoDUNE-DP est une expérience de prototypage qui a pour but de tester la technologie DLArTPC à grande échelle dans le but de l'utiliser dans la future expérience de physique des neutrinos DUNE, prévue fin 2026 aux USA. Le travail de cette thèse s'oriente dans une premier temps autour de la préparation des éléments de détection et d'amplification du démonstrateur de 300t de WA105/ProtoDUNE-DP, dont la mise en route a été effectuée fin août 2019. Dans un second temps, la thèse s'oriente autour de l'analyse des résultats du prototype de 4t de WA105, opéré en 2017.\\

La technologie DLArTPC est une variante de la technologie LArTPC permettant une amplification du signal dans une fine couche d'argon gazeux en haut du volume de détection. Les amplificateurs d'électrons (LEM) sont des plaques de PCB de 50x50cm2 épais de 1mm, percés de 400k trous de 500 microns. Une partie du travail de cette thèse à consisté à simulé la dérive des électrons à travers ces amplificateurs. Une autre partie de cette thèse à consisté à mesurer les caractéristiques importantes (épaisseur, tenue en tension) de ces amplificateurs.\\

Le gain est la caractéristique principal d'une DLArTPC, et il a été étudié dans le prototype de 4t. Des comparaisons sont effectuées avec les résultats d'un prototype de 3L datant de 2014, et un programme de reconstruction de trace dédié a été développé pour traiter certains événements bruités.}															%% Résumé en français / abstract in french

\newcommand{\keywordsEN}{Neutrino physics, WA105, DLArTPC, LEM, DUNE}														%% Mots clés en anglais, séprarés par des , / Keywords in english, separated by ,
\newcommand{\abstractEN}{The WA105/ProtoDUNE-DP is a prototyping experiment aiming to test the DLArTPC technology at large scale with the goal to use it in the DUNE neutrino physics experiment. Scheduled for the end of 2026 in the USA, DUNE will measure the neutrinos mass ordering and the CP symetry violation in the leptonic sector. This thesis starts with a part centered around the preparation of the detectors and amplificators of the \SI{300}{\tonne} demonstrator of WA105/ProtoDUNE-DP, which commissionning was done at the end of August 2019. The second part of this thesis is about the analysis of the data taken with the \SI{4}{\tonne} prototype of WA105, operated in 2017.\\

The DLArTPC technology is a variation of the LArTPC technology, allowing for an amplification of the signal in a thin layer of gaseous argon on top of the detector volume. The Large Electron Amplifiers (LEMs) are PCB plates of $50\times\SI{50}{\centi\meter\squared}$ with a thickness of \SI{1}{\milli\meter} and pierced with \numprint{400000} holes of \SI{500}{\micro\meter}. A part of this thesis work is the simulation the drift of electrons through those LEMs. Another part is the measurement of important caractericstics (thickness, voltage stability) of those LEMs.\\

The gain is the main caracteristic of a DLArTPC. It has been studied in the \SI{4}{t} prototype, comparisons with results from a \SI{3}{\liter} prototype from 2014 are done. A reconstruction program dedicated to finding tracks in noisy events was developped and used.}															%% Résumé en anglais / abstract in english